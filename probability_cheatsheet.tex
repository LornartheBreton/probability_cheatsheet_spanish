\documentclass[12,landscape]{article}
\usepackage[spanish]{babel}
\usepackage{multicol}
\usepackage{calc}
\usepackage{ifthen}
\usepackage[landscape]{geometry}
\usepackage{graphicx}
\usepackage{amsmath, amssymb, amsthm}
\usepackage{latexsym, marvosym}
\usepackage{pifont}
\usepackage{lscape}
\usepackage{graphicx}
\usepackage{array}
\usepackage{booktabs}
\usepackage[bottom]{footmisc}
\usepackage{tikz}
\usetikzlibrary{shapes}
\usepackage{pdfpages}
\usepackage{wrapfig}
\usepackage{enumitem}
\setlist[description]{leftmargin=0pt}
\usepackage{xfrac}
\usepackage[pdftex,
            pdfauthor={Héctor G. T. Torres},
            pdftitle={Probability Cheatsheet},
            pdfsubject={A cheatsheet pdf and reference guide originally made for Stat 110, Harvard's Introduction to Probability course. Formulas and equations for your statistics class.},
            pdfkeywords={probability} {statistics} {cheatsheet} {pdf} {cheat} {sheet} {formulas} {equations}
            ]{hyperref}
\usepackage[
            open,
            openlevel=2
            ]{bookmark}
\usepackage{relsize}
\usepackage{rotating}

 \newcommand\independent{\protect\mathpalette{\protect\independenT}{\perp}}
    \def\independenT#1#2{\mathrel{\setbox0\hbox{$#1#2$}%
    \copy0\kern-\wd0\mkern4mu\box0}} 
            
\newcommand{\noin}{\noindent}    
\newcommand{\logit}{\textrm{logit}} 
\newcommand{\var}{\textrm{Var}}
\newcommand{\cov}{\textrm{Cov}} 
\newcommand{\corr}{\textrm{Corr}} 
\newcommand{\N}{\mathcal{N}}
\newcommand{\Bern}{\textrm{Bern}}
\newcommand{\Bin}{\textrm{Bin}}
\newcommand{\Beta}{\textrm{Beta}}
\newcommand{\Gam}{\textrm{Gamma}}
\newcommand{\Expo}{\textrm{Expo}}
\newcommand{\Pois}{\textrm{Pois}}
\newcommand{\Unif}{\textrm{Unif}}
\newcommand{\Geom}{\textrm{Geom}}
\newcommand{\NBin}{\textrm{NBin}}
\newcommand{\Hypergeometric}{\textrm{HGeom}}
\newcommand{\HGeom}{\textrm{HGeom}}
\newcommand{\Mult}{\textrm{Mult}}
\newcommand{\Cauchy}{\textrm{C}}
\newcommand{\Wei}{\textrm{Wei}}
\newcommand{\Par}{\textrm{Par}}

\geometry{top=.4in,left=.2in,right=.2in,bottom=.4in}

\pagestyle{empty}
\makeatletter
\renewcommand{\section}{\@startsection{section}{1}{0mm}%
                                {-1ex plus -.5ex minus -.2ex}%
                                {0.5ex plus .2ex}%x
                                {\normalfont\large\bfseries}}
\renewcommand{\subsection}{\@startsection{subsection}{2}{0mm}%
                                {-1explus -.5ex minus -.2ex}%
                                {0.5ex plus .2ex}%
                                {\normalfont\normalsize\bfseries}}
\renewcommand{\subsubsection}{\@startsection{subsubsection}{3}{0mm}%
                                {-1ex plus -.5ex minus -.2ex}%
                                {1ex plus .2ex}%
                                {\normalfont\small\bfseries}}
\makeatother

\setcounter{secnumdepth}{0}

\setlength{\parindent}{0pt}
\setlength{\parskip}{0pt plus 0.5ex}

% -----------------------------------------------------------------------

\usepackage{titlesec}

\titleformat{\section}
{\color{black}\normalfont\large\bfseries}
{\color{black}\thesection}{1em}{}
\titleformat{\subsection}
{\color{black}\normalfont\normalsize\bfseries}
{\color{black}\thesection}{1em}{}
% Comment out the above 5 lines for black and white

\begin{document}

\raggedright
\footnotesize
\begin{multicols*}{3}

% multicol parameters
% These lengths are set only within the two main columns
%\setlength{\columnseprule}{0.25pt}
\setlength{\premulticols}{1pt}
\setlength{\postmulticols}{1pt}
\setlength{\multicolsep}{1pt}
\setlength{\columnsep}{2pt}

%%%%%%%%%%%%%%%%%%%%%%%%%%%%%%%%%%%%
%%% TITLE
%%%%%%%%%%%%%%%%%%%%%%%%%%%%%%%%%%%%

\begin{center}
    {\color{black} \Large{\textbf{Cálculo de Probabilidades I}}} \\
   % {\Large{\textbf{Probability Cheatsheet}}} \\
    % comment out line with \color{blue} and uncomment above line for b&w
\end{center}

%%%%%%%%%%%%%%%%%%%%%%%%%%%%%%%%%%%%
%%% ATTRIBUTIONS
%%%%%%%%%%%%%%%%%%%%%%%%%%%%%%%%%%%%

\scriptsize

Compilado, traducido y adaptado al curso Cálculo de Probabilidades I del ITAM por Héctor G. T. Torres, basado en el trabajo de William Chen y Joe Blitzstein, con contribuciones de Sebastian Chiu, Yuan Jiang, Yuqi Hou, y Jessy Hwang. Material basado en el libro Introduction to Probability (\url{http://bit.ly/introprobability}) de Blitztein/Hwang. Licenciado bajo \texttt{\href{http://creativecommons.org/licenses/by-nc-sa/4.0/}{CC BY-NC-SA 4.0}}. 

\begin{center}
    Compilado el \today
\end{center}

% Cheatsheet format from
% http://www.stdout.org/$\sim$winston/latex/

%%%%%%%%%%%%%%%%%%%%%%%%%%%%%%%%%%%%
%%% BEGIN CHEATSHEET
%%%%%%%%%%%%%%%%%%%%%%%%%%%%%%%%%%%%


\section{Conteo}\smallskip \hrule height 2pt \smallskip

% \subsection{Set Theory}

% \begin{description}
%     \item[Sets and Subsets] - A set is a collection of distinct objects. $A$ is a subset of $B$ if every element of $A$ is also included in $B$.
%     \item[Empty Set] - The empty set, denoted $\emptyset$, is the set that contains nothing.
%     \item[Set Notation] - Note that ${\bf {\bf A}} \cup {\bf B}$, ${\bf A} \cap {\bf B}$, and ${\bf A^c}$ are all sets too.
%     \begin{description}
%         \item[Union] - ${\bf A} \cup {\bf B}$ (read \emph{{\bf A} union {\bf B}}) means ${\bf A}\ or\ {\bf B}$
%         \item[Intersection] - ${\bf A} \cap {\bf B}$ (read \emph{{\bf A} intersect {\bf B}}) means ${\bf A}\ and \ {\bf B}$
%         \item[Complement] - ${\bf A^c}$ (read \emph{{\bf A} complement}) occurs whenever ${\bf A}$ does not occur
%     \end{description}
%     \item[Disjoint Sets] - Two sets are disjoint if their intersection is the empty set (e.g. they don't overlap).
%     \item[Partition] - A set of subsets ${\bf A}_1, {\bf A}_2, {\bf A}_3, ... {\bf A}_n$ partition a space if they are disjoint and cover all possible outcomes (e.g. their union is the entire set). A simple case of a partitioning set of subsets is ${\bf A}, {\bf A^c}$
%         \item[Principle of Inclusion-Exclusion] - Helps you find the probabilities of unions of events. 
%         \[ P ({\bf A} \cup {\bf B}) = P({\bf A}) + P({\bf B}) - P({\bf A} \cap {\bf B}) \]
%         \[P(\textnormal{Union of many events}) = \textnormal{Singles} - \textnormal{Doubles} + \textnormal{Triples} - \textnormal{Quadruples} \dots\]


% \end{description}
    \subsection{Regla de la Multiplicación} 
    \begin{minipage}{\linewidth}
            \centering
            \includegraphics[width=2in]{figures/icecream.pdf}
        \end{minipage}

 Digamos que tenemos un experimento compuesto (un experimento con varios componentes). Si el primer componente tiene $n_1$ resultados posibles, el segundo componente tiene $n_2$ resultados posibles, \dots, y el componente $r$ tiene $n_r$ resultados posibles, entonces habrá en total $n_1n_2 \dots n_r$ resultados posibles para todo el experimento.
 
\subsection{Tabla de Muestreo}
   \begin{minipage}{\linewidth}
            \centering
             \includegraphics[width=1.2in]{figures/jar.pdf}
        \end{minipage}
      %  \begin{center}
    Una tabla de muestreo da el número de muestras posibles de tamaño $k$ de una población de tamaño $n$, bajo varios supuestos sobre cómo se condujo ese muestreo.
        %\begin{table}[H]
        \begin{center}
                \setlength{\extrarowheight}{7pt}
            \begin{tabular}{r|cc}
                 & \textbf{Importa el orden} & \textbf{No importa} \\ \hline
                \textbf{Con Reemplazo} & $\displaystyle n^k$ & $\displaystyle{n+k-1 \choose k}$ \\
                \textbf{Sin Reemplazo} & $\displaystyle\frac{n!}{(n - k)!}$ & $\displaystyle{n \choose k}$
            \end{tabular}
        \end{center}
        %\end{table}
    % \item[Experiments/Outcomes] - An experiment generates an outcome from a pre-determined list. For example, a dice roll generates outcomes in the set $\{1, 2, 3, 4, 5, 6\}$
    % \item[Sample Space] - The sample space, denoted $\Omega$, is the set of possible outcomes. Note that the probability of this event is 1, since something in the sample space will always occur.
    % \item[Event] - An event is a subset of the sample space, or a collection of possible outcomes of an experiment. We say that the event has occurred if any of the outcomes in the event have happened.
    \subsection{Definición Ingenua de probabilidad}  {Si todos los resultados son equiprobables}, la probabilidad de que ocurra un evento $A$ es:
        \[P_{\textrm{clásica}}(A) = \frac{\textnormal{número de resultados favorables a $A$}}{\textnormal{número de resultados}}\]

\section{Pensando Condicionalmente} \smallskip \hrule height 2pt \smallskip

% \subsection{Set Theory and Statistics}
% %To understand probability it helps to understand basic set theory. An \emph{event} is a set in that it is a collection of possible outcomes of an experiment (or a subset of the sample space). With set theory we can talk about things like unions, intersections, or complements of events.

% \begin{description}
%     \item[Experiments/Outcomes] - An experiment generates an outcome from a pre-determined list. For example, a dice roll generates outcomes in the set $\{1, 2, 3, 4, 5, 6\}$
%     \item[Sample Space] - The sample space, denoted $\Omega$, is the set of possible outcomes. Note that the probability of this event is 1, since something in the sample space will always occur.
%     \item[Event] - An event is a subset of the sample space, or a collection of possible outcomes of an experiment. We say that the event has occurred if any of the outcomes in the event have happened.
% \end{description}

%\subsection{Disjointness Versus Independence}
\subsection{Independencia}

    \begin{description}
        % \item[Disjoint Events] - ${\bf A}$ and ${\bf B}$ are disjoint when they cannot happen simultaneously, or
        %   \begin{align*}
        %     P({\bf A} \cap {\bf B}) &= 0\\
        %     {\bf A} \cap {\bf B} &= \emptyset
        %   \end{align*}
        \item[Eventos Independientes] $A$ y $B$ son independientes si el saber que $A$ ocurrió no me da información sobre si $B$ ocurrió. Formalmente, los eventos $A$ y $B$ (ambos con una probabilidad mayor a $0$) son independientes sí y sólo sí se cumple una de las siguientes equivalencias.
           \begin{align*} 
            P({A}\cap { B}) &= P({A})P({B}) \\
            P({ A}|{ B}) &= P({A})\\
            P(B|A) &= P(B)
           \end{align*}
        \item[Independencia Condicional] $A$ y $B$ son condicionalmente independientes dado $C$ si $P(A \cap B | C) = P(A|C)P(B|C)$. \textbf{Nota:} la independencia condicional \textbf{no} implica independencia, y la independencia \textbf{no} implica independencia condicional.
    \end{description}
    
\subsection{Uniones, Intersecciones, y Complementos}

    \begin{description}

        \item[Leyes de De Morgan] Identidades muy útiles que ayudan a calcular las probabilidades de uniones al relacionarlas con intersecciones y viceversa. Los resultados se mantienen con más de dos conjuntos.
           \begin{align*} 
        ({A} \cup { B})^c = {A^c} \cap { B^c} \\
        ({A} \cap {B})^c = { A^c} \cup { B^c}
           \end{align*} 
                  
        % \item[Complements] - The following are true.
        %    \begin{align*} 
        %      {\bf A} \cup {\bf A}^c &= \Omega \\
        %      {\bf A} \cap {\bf A}^c &= \emptyset\\
        %      P({\bf A}) &= 1 -  P({\bf A}^c)
        %    \end{align*} 

    \end{description}

\subsection{Probabilidades Conjuntas, Marginales, y Condicionales}

    \begin{description}
        \item[Probabilidad Conjunta] $P (A \cap B)$ ó $P (A,B)$ -- Probabilidades de $A$ y $B$.
        \item[Probabilidad Marginal (Incondicional)] $P(A)$ -- Probabilidad de $A$.
        \item[Probabilidad Condicional] $P(A | B) = \frac{P(A,B)}{P(B)}$ -- Probabilidad de A, dado que ocurrió B.
        \item[La Probabilidad Condicional \emph{es} Probabilidad] $P(A|B)$ es una función de probabilidad para cualquier $B$ dada. Cualquier teorema que se da para la probabilidad también se da para la probabilidad condicional.
    \end{description}

\subsection{Probabilidad de una Intersección o Unión}
\textbf{Intersecciones por Condicionamiento}
\begin{align*} 
    P(A,B) &= P(A)P(B|A) \\
   P(A,B,C) &= P(A)P(B|A)P(C|A,B) 
   \end{align*}
   \textbf{Uniones por Inclusión-Exclusión}
   \begin{align*} 
    P(A \cup B) &= P(A) + P(B) - P(A \cap B) \\
  P(A \cup B \cup C) &= P(A) + P(B) + P(C) \\
&\quad - P(A \cap B) - P(A \cap C) - P(B \cap C) \\
&\quad + P(A \cap B \cap C).
   \end{align*}


%\subsection{Simpson's Paradox}
%\begin{minipage}{\linewidth}
%            \centering
%\includegraphics[width=2in]{figures/SimpsonsParadox.pdf}
%        \end{minipage}
% It is possible to have
%\[P(A\mid B,C) < P(A\mid B^c, C) \textnormal{ and } P(A\mid B, C^c) < P(A \mid B^c, C^c)\]
%\[ \textnormal{yet also } P(A\mid B) > P(A \mid B^c).\]

\subsection{Teorema de la Probabilidad Total (TPT)}
Sea $B_1, B_2, B_3, \dots, B_n$ una \emph{partición} del espacio muestral (es decir, que son \emph{disconjuntas} y su unión compone al espacio muestral en su totalidad). Entonces, se dá que:

    \begin{align*}
P(A) = P(A|B_1)P(B_1) + P(A|B_2)P(B_2) + \dots + P(A|B_n)P(B_n)\\
P(A) = P(A \cap B_1) + P(A \cap B_2) + \dots + P(A \cap B_n)
    \end{align*}

Si las particiones son $B$ y $B^c$, podemos resumirlo como:

    \begin{align*}
P(A) = P(A|B)P(B) + P(A|B^c)P(B^c)      
    \end{align*}

\subsection{Regla de Bayes}
\textbf{Regla de Bayes}
\[P(A|B) = \frac{P(B|A)P(A)}{P(B)}\]

También podemos escribir

$$P(A|B,C) = \frac{P(A,B,C)}{P(B,C)} = \frac{P(B,C|A)P(A|C)}{P(B,C)}$$

\textbf{Un caso raro de la Regla de Bayes}
\[\frac{P(A|B)}{P(A^c|B)} = \frac{P(B|A)}{P(B|A^c)} \frac{P(A)}{P(A^c)}\]

Se puede interpretar como que la \emph{probabilidad posterior} de $A$ es igual a \emph{la razón de probabilidades} por la \emph{probabilidad anterior}.

\section{Variables Aleatorias y sus Distribuciones}\smallskip \hrule height 2pt \smallskip
   
\subsection{FMP, FDA, e Independencia}
\begin{description}

\item[Función de Masa de Probabilidad (FMP, PMF en inglés)] Da la probabilidad que una variable aleatoria \emph{discreta} tenga el valor $x$.
\[ p_X(x) = P(X=x)\]
\begin{minipage}{\linewidth}
   \centering 
   \includegraphics[width=2in]{figures/Binpmf.pdf}
\end{minipage}

\item[Función de Probabilidad Acumulada (FDA, CDF en inglés)] 
Da la probabilidad de que una variable aleatoria sea igual o menor a $X.$
\[F_X(x) = P(X \leq x)\]
\begin{minipage}{\linewidth}
            \centering
\includegraphics[width=2in]{figures/Bincdf.pdf}
        \end{minipage}

La FDA es una función creciente y contínua a la derecha que cumple con
\[\lim_{x\to -\infty} {F_X(x)} = 0\]
\[\lim_{x \to \infty} F_X(x) = 1\]

\end{description}

\section{Esperanza e Indicadoras}\smallskip \hrule height 2pt \smallskip


\subsection{Esperanza}
\begin{description}
\item[Esperanza] (alias~\emph{media}, \emph{valor esperado}, o \emph{promedio}) es un promedio ponderado de los posibles resultados de una variable aleatoria, ponderados por la probabilidad de cada evento. Matemáticamente, si $x_1, x_2, x_3, \dots$ son los distintos valores que puede tomar la v.a. $X$, entonces la esperanza de $X$ es
\begin{center}
$E(X) = \sum\limits_{i}x_iP(X=x_i)$
\end{center}

\begin{minipage}{\linewidth}
            \centering
\includegraphics[width=2in]{figures/linearity.pdf}
        \end{minipage}


\item[Linealidad de la Esperanza] Para cualquier v.a. $X$ y $Y$, y constantes $a,b,c,$ 
\[E(aX + bY + c) = aE(X) + bE(Y) + c \]

\item[La misma distribución implica la misma media] Si $X$ y $Y$ siguen la misma distribución, entonces $E(X)=E(Y)$ y, en la generalidad, 
$$E(g(X)) = E(g(Y))$$


\item[Esperanza Condicional] se define como la Esperanza, solo que condicionada en cualquier evento $A$. \begin{center}
$\ E(X | A) = \sum\limits_{x}xP(X=x | A)$
\end{center}

\end{description}

\subsection{Variables Aleatorias Indicadoras}
\begin{description}
\item[Variable Aleatoria Indicadora] Es una variable aleatoria que toma el valor de 1 o 0. Siempre es indicativa de algún evento: si el evento ocurre, toma el valor de 1; sino, toma el valor de 0. Son útiles para muchos problemas para contar cuántos eventos de algún tipo ocurrieron en cierto tiempo. Se escribe \[
I_A =
 \begin{cases}
   1 & \text{si $A$ ocurre,} \\
   0 & \text{si $A$ no ocurre.}
  \end{cases}
\]

Nótese que $I_A^2 = I_A, I_A I_B = I_{A \cap B}, $ and $I_{A \cup B} = I_A + I_B - I_A I_B$. 
\item[Distribución] $I_A \sim \Bern(p)$ donde $p = P(A)$.
\item[Puente Fundamental] La esperanza de que el indicador del evento $A$ es la probabilidad del evento $A$: $E(I_A) = P(A)$. 
\end{description}

\subsection{Varianza y Desviación Estándar}
\[\var(X) = E \left(X - E(X)\right)^2 = E(X^2) - (E(X))^2\]
\[\textrm{SD}(X) = \sqrt{\var(X)}\]

\section{V.A.s Contínuas, TEI (LOTUS), UoU}\smallskip \hrule height 2pt \smallskip

\subsection{Variables Aleatorias Contínuas (V.A.C.s)}
\begin{description}
\item[Probabilidad de que una V.A.C. se encuentre en un intérvalo] Es la diferencia en los valores de la FDA (o usa la FDP que describiremos en un momento).
\[P(a \leq X \leq b) = P(X \leq b) - P(X \leq a) = F_X(b) - F_X(a)\]

Para cualquier $X \sim \N(\mu,\sigma^2)$, esto se convierte en
\begin{align*}
P(a\leq X\leq b)&=\Phi \left(\frac{b-\mu }{\sigma } \right) - \Phi \left( \frac{a-\mu }{\sigma } \right)
\end{align*}

\item[Función de Densidad de Probabilidad (FDP, PDF en inglés)] la FDP $f$ es la derivada de la FDA $F$.
\[ F'(x) = f(x) \]
Una FDP es no-negativa e integra a 1. Por el Teorema Fundamental del Cálculo, podemos convertir una FDP en una FDA integrando de la siguiente manera:
\begin{align*} 
    F(x) &=  \int_{-\infty}^x f(t)dt  
   \end{align*}
   \begin{minipage}{\linewidth}
            \centering
\includegraphics[width=2in]{figures/Logisticpdfcdf.pdf}
        \end{minipage}
   Para calcular la probabilidad de que una VAC tome un valor en un intervalo, tenemos que integrar la FDP sobre ese intervalo.
      \begin{align*} 
    F(b) - F(a)  &=  \int^b_a f(x)dx
       \end{align*}
   
\item[Esperanza de una VAC] De forma análoga al caso discreto, donde sumas $x$ por la FMP, para las VACs integras $x$ por la FDP.
\[E(X) = \int^\infty_{-\infty}xf(x)dx \]
% Review: Expected value is \emph{linear}. This means that for \emph{any} random variables $X$ and $Y$ and any constants $a, b, c$, the following is true:
% \[E(aX + bY + c) = aE(X) + bE(Y) + c\]
\end{description}


\label{tei}
\subsection{TEI (LOTUS)}
\begin{description}
\item[Esperanza de una función de una v.a.]
Así se define la Esperanza de X
\[E(X) = \sum_x xP(X=x) \textnormal{ (Si $X$ es discreta)}\]
\[E(X) = \int^\infty_{-\infty}xf(x)dx  \textnormal{ (Si $X$ es contínua)}\]
El \textbf{Teorema del Estadístico Inconsciente (LOTUS)} indica que puedes encontrar la Esperanza de una \emph{función de una v.a.}, $g(x)$ de una manera similar, al reemplazar la $x$ en frente de la FMP/FDP por $g(x)$, pero menteniendo la FMP/FDP de $X$:
\[E(g(X)) = \sum_x g(x)P(X=x) \textnormal{ (Si $X$ es discreta)}\]
\[E(g(X)) = \int^\infty_{-\infty}g(x)f(x)dx \textnormal{ (Si $X$ es contínua)}\]
\item[¿Qué es la función de una variable aleatoria?] Una función de una v.a. es también una v.a. Por ejemplo, si $X$ es el número de bicicletas que ves en una hora, entonces $g(X) = 2X$ es el número de llantas de bicicletas que ves en una hora, y $h(X) = {X \choose 2} = \frac{X(X-1)}{2}$ es el número de \emph{pares} de bicicletas tal que ves esas dos bicicletas en la misma hora.
\item[¿Cuál es el punto de esto?] El punto es que \emph{no} necesitas conocer la FMP/FDP de $g(X)$ para calcular su esperanza. Lo único que necesitas es la FMP/FDP de $X$.
\end{description}

\subsection{Universalidad de la Uniforme (UoU)} Cuando metes cualquier VAC en su propia FDA, obtienes una v.a. Uniforme(0,1). Cuando metes una v.a. Uniforme(0,1) ~en una FDA inversa, obtienes una v.a. con esa FDA. Por ejemplo, digamos que una v.a. $X$ tiene una FDA
    \[ F(x) = 1 - e^{-x}, \textrm{ para $x>0$} \]
    Por la UoU, si ponemos $X$ en esta función, obtendremos una variable aleatoria distribuida de manera uniforme.
    \[ F(X) = 1 - e^{-X} \sim \textrm{Unif}(0,1)\]
    De forma similar, si $U \sim \textrm {Unif}(0,1)$ entonces la inversa $F^{-1}(U)$ tiene la FDA $F$. El punto es que para cualquier variable aleatoria contínua $X$, podemos transformarla y destransformarla en en una variable aleatoria Uniforme utilizando su FDA.

\section{Momentos y FGMs}\smallskip \hrule height 2pt \smallskip

\subsection{Momentos}

Los Momentos describen la forma de una distribución. Sea $X$ con una media $\mu$ y una desviación estándar $\sigma$, y que $Z=\frac{(X-\mu)}{\sigma}$ sea la versión \emph{estandarizada} de $X$. El $k$ momento de $X$ es $\mu_k = E(X^k)$ y el $k$ momento estandarizado de $X$ es $m_k = E(Z^k)$. La media, varianza, asimetría (skew) y kurtosis son resúmenes importantes de la forma de una distribución.
    \begin{description}
        \item[Mean] $E(X) = \mu_1 $
        \item[Variance] $\var(X) = \mu_2 - \mu_1^2$
        \item[Skewness] $\textrm{Skew}(X) = m_3$
      \item[Kurtosis] $\textrm{Kurt}(X) = m_4 - 3$
    \end{description}

\subsection{Funciones Generadoras de Momentos (FGMs, MGFs en inglés)}

\begin{description}
    \item[FGM] Para cualquier variable aleatoria $X$, la función
        \[ M_X(t) = E(e^{tX}) \]
        es la \textbf{función generadora de momentos (FGM)} de $X$, si existe para toda $t$ en un intérvalo abierto que contenga $0$. La variable $t$ pudo haberse llamado $u$, o $v$. Es un atajo que nos permite trabajar con la \emph{función} $M_X$ en vez de la \emph{secuencia} de momentos.
                
            \item[¿Por qué se llama Función Generadora de Momentos?] Porque la derivada de orden $k$ de la función generadora de momentos, evaludada en $0$, es el momento $k$ de $X$.
    \[\mu_k = E(X^k) = M_X^{(k)}(0)\]

    Esto se demuestra con la expansión de Taylor de $e^{tX}$, ya que
    \[M_X(t) = E(e^{tX}) = \sum_{k=0}^\infty \frac{E(X^k)t^k}{k!} = \sum_{k=0}^\infty \frac{\mu_k t^k}{k!} \]

    \item[FGM de funciones lineales] Si tenemos que $Y = aX + b$, entonces
        \[M_Y(t) = E(e^{t(aX + b)}) =  e^{bt}E(e^{(at)X}) = e^{bt}M_X(at)\]

    \item[Unicidad] \emph{Si existe, la FGM determina de manera única la distribución.} Esto significa que, para cualesquiera variables aleatorias $X$ y $Y$, éstas se distribuyen igual (sus FMPs/FDAs son las mismas) sí y sólo si sus FGMs son las mismas.

    \item[Sumando v.a.s independientes multiplicando sus FGMs] Si $X$ y $Y$ son independientes, entonces
    \begin{align*}
        M_{X+Y}(t) &= E(e^{t(X + Y)}) = E(e^{tX})E(e^{tY}) = M_X(t) \cdot M_Y(t) 
    \end{align*}
    La FGM de la suma de dos variables aleatorias es el producto de la FGM de esas dos variables aleatorias.
\end{description}

\section{Juegos con FDPs y FDAs}\smallskip \hrule height 2pt \smallskip

\subsection{Distribuciones Conjuntas}
La \textbf{FDA conjunta}  de $X$ y $Y$ es
$$F(x,y)=P(X \leq x, Y \leq y)$$
Si son variables discretas, $X$ y $Y$ tienen una \textbf{FMP conjunta}
$$p_{X,Y}(x,y) = P(X=x,Y=y).$$ 
En el caso contínuo, tienen una \textbf{FDP conjunta}
\[f_{X,Y}(x,y) = \frac{\partial^2}{\partial x \partial y} F_{X,Y}(x,y).\]
La FMP/FDP conjunta debe de ser no-negativa y sumar/integrar a 1.
\begin{minipage}{\linewidth}
            \centering
\includegraphics[width=1.0in]{figures/jointPDF.pdf}
        \end{minipage}
        
\subsection{Distribuciones Condicionales}
\textbf{Condicionales y Regla de Bayes para v.a.s discretas}
\[P(Y=y|X=x) = \frac{P(X=x, Y=y)}{P(X=x)} = \frac{P(X=x|Y=y)P(Y=y)}{P(X=x)}\]
\textbf{Condicionales y Regla de Bayes para v.a. s contínuas}
\[f_{Y|X}(y|x) = \frac{f_{X,Y}(x, y)}{f_X(x)} = \frac{f_{X|Y}(x|y)f_Y(y)}{f_X(x)}\]
\textbf{Regla de Bayes Híbrida}
\[f_X(x|A) = \frac{P(A | X = x)f_X(x)}{P(A)}\]

\subsection{Distribuciones Marginales}
Para encontrar la distribución de una (o más) variables aleatorias a partir de una sola FDP/FMP conjunta, debes de sumar/integrar sobre las variables aleatorias que quieres eliminar.

\textbf{FMP marginal de FMP conjunta}
\[P(X = x) = \sum_y P(X=x, Y=y)\]
\textbf{FDP marginal de FDP conjunta}
\[f_X(x) = \int_{-\infty}^\infty f_{X, Y}(x, y) dy\]


\subsection{Independencia de Variables Aleatorias}
Las variables aleatorias $X$ y $Y$ son independientes sí y solo sí se cumplen cualesquiera de las siguientes condiciones.

\begin{itemize}
    \item -lmm
    \item La FDA conjunta es el producto de las FDAs marginales
    \item La FMP/FDP conjunta es el producto de las FMP/FDP marginales
    \item La distribución condicional de $Y$ dado $X$ es la distribución maginal de $Y$
\end{itemize}
 En cuyo caso, escribe $X \independent Y$ para denotar que $X$ y $Y$ son independientes.

%\subsection{Multivariate LOTUS}
%LOTUS in more than one dimension is analogous to the 1D LOTUS.
%For discrete random variables:
%\[E(g(X, Y)) = \sum_x\sum_yg(x, y)P(X=x, Y=y)\]
%For continuous random variables:
%\[E(g(X, Y)) = \int_{-\infty}^{\infty}\int_{-\infty}^{\infty}g(x, y)f_{X,Y}(x, y)dxdy\]
%
\section{Covarianza}\smallskip \hrule height 2pt \smallskip

\subsection{Covarianza y Correlación}
\begin{description}
\item [Covarianza] es la análoga de la Varianza, pero para dos variables aleatorias.
    \[\cov(X, Y) = E\left((X - E(X))(Y - E(Y))\right) = E(XY) - E(X)E(Y)\]
    Nótese que
    \[\cov(X, X) = E(X^2) - (E(X))^2 =  \var(X)\]
\item [Correlación] es una versión estandarizada de la Covarianza, tal que siempre se encuentra entre $-1$ y $1$.
    \[\corr(X, Y) = \frac{\cov(X, Y)}{\sqrt{\var(X)\var(Y)}} \]
    
\item [Covarianza e Independencia] Si dos variables aleatorias son independientes, entonces no están correlacionadas. Sin embargo, lo inverso no es necesariamente verdad (ej. digamos que $X \sim \N(0,1)$ y $Y = X^2$).
    \begin{align*}
    	X \independent Y &\longrightarrow \cov(X, Y) = 0 \longrightarrow E(XY) = E(X)E(Y)
    \end{align*}
%, except in the case of Multivariate Normal, where uncorrelated \emph{does} imply independence.
\item [Covarianza y Varianza] Se puede calcular la Varianza de una suma de la siguiente manera
    \begin{align*}
        %\cov(X, X) &= \var(X) \\
        \var(X + Y) &= \var(X) + \var(Y) + 2\cov(X, Y) \\
        \var(X_1 + X_2 + \dots + X_n ) &= \sum_{i = 1}^{n}\var(X_i) + 2\sum_{i < j} \cov(X_i, X_j)
    \end{align*}
    Si $X$ y $Y$ son independientes, entonces tienen una Covarianza de $0$, entonces
    \[X \independent Y \Longrightarrow \var(X + Y) = \var(X) + \var(Y)\]
    Si $X_1,X_2, \dots, X_n$ se distribuyen de manera idéntica y tienen las mismas relaciones de covarianza (lo cual ocurre muchas veces por \textbf{simetría}), entonces
    \[\var(X_1 + X_2 + \dots + X_n ) = n\var(X_1) + 2{n \choose 2}\cov(X_1, X_2)\]
    
\item [Propiedades de la Covarianza]  Para las variables aleatorias $W,X,Y,Z$ y las constantes $a, b$: 
    \begin{align*}
    	\cov(X, Y) &= \cov(Y, X) \\
        \cov(X + a, Y + b) &= \cov(X, Y) \\
        \cov(aX, bY) &= ab\cov(X, Y) \\
        \cov(W + X, Y + Z) &= \cov(W, Y) + \cov(W, Z) + \cov(X, Y)\\
        &\quad + \cov(X, Z)
    \end{align*}
\item [Correlación es invariante en cuando a ubicación y escala] Para cualquiera constantes $a,b,c,d$ con $a$ y $c$ diferentes de $0$.
    \begin{align*}
        \corr(aX + b, cY + d) &= \corr(X, Y) 
    \end{align*}
\end{description}

%\subsection{Transformations}
%\begin{description}
%    \label{one variable transformations}
%    \item[One Variable Transformations] Let's say that we have a random variable $X$ with PDF $f_X(x)$, but we are also interested in some function of $X$. We call this function $Y = g(X)$. Also let $y=g(x)$. If $g$ is differentiable and strictly increasing (or strictly decreasing), then the PDF of $Y$ is
%    \[f_Y(y) = f_X(x)\left|\frac{dx}{dy}\right| =  f_X(g^{-1}(y))\left|\frac{d}{dy}g^{-1}(y)\right|\]
%    The derivative of the inverse transformation is called the \textbf{Jacobian}.
%
%
%     \item[Two Variable Transformations] Similarly, let's say we know the joint PDF of $U$ and $V$ but are also interested in the random vector $(X, Y)$ defined by $(X, Y) = g(U, V)$. Let 
%       $$  \frac{\partial (u,v)}{\partial (x,y)}  = \begin{pmatrix} 
%              \frac{\partial u}{\partial x} &  \frac{\partial u}{\partial y} \\
%           \frac{\partial v}{\partial x} & \frac{\partial v}{\partial y}   \\
%        \end{pmatrix}$$
%     be the \textbf{Jacobian matrix}. If the entries in this matrix exist and are continuous, and the determinant of the matrix is never $0$, then
%     \[f_{X,Y}(x, y) = f_{U,V}(u,v) \left|\left|   \frac{\partial (u,v)}{\partial (x,y)}\right| \right| \]
%   The inner bars tells us to take the matrix's determinant, and the outer bars tell us to take the absolute value.  In a $2 \times 2$ matrix, 
%     \[ \left| \left|
%     \begin{array}{ccc}
%         a & b \\
%         c & d
%     \end{array}
%     \right| \right| = |ad - bc|\]
%
%\end{description}
%
%\label{convolutions}
%\subsection{Convolutions}
%\begin{description}
%    \item[Convolution Integral] If you want to find the PDF of the sum of two independent CRVs $X$ and $Y$, you can do the following integral:
%        \[f_{X+Y}(t)=\int_{-\infty}^\infty f_X(x)f_Y(t-x)dx\]
%    \item[Example] Let $X,Y \sim \N(0,1)$ be i.i.d. Then for each fixed $t$,\[f_{X+Y}(t)=\int_{-\infty}^\infty \frac{1}{\sqrt{2\pi}}e^{-x^2/2} \frac{1}{\sqrt{2\pi}}e^{-(t-x)^2/2} dx\]
%By completing the square and using the fact that a Normal PDF integrates to $1$, this works out to $f_{X+Y}(t)$ being the $\N(0,2)$ PDF.
%\end{description}
%
\section{Proceso Poisson}\smallskip \hrule height 2pt \smallskip

\begin{description}
\item[¿Qué es?] Es una \emph{story} que liga a varias distribuciones, en particular la Esponencial y la Poisson.

\item[Definición] Tenemos un \textbf{Proceso Poisson} de una tasa de $\lambda$ ocurrencias por unidar temporal si se dan las siguientes condiciones:
\begin{enumerate}
    \item El número de ocurrencias en un intervalo de tiempo de longitud $t$ se distribuye $\Pois(\lambda t)$
    \item El número de ocurrencias en intervalos disconjuntos son independientes entre sí.
\end{enumerate}
Por ejemplo, el número de ocurrencias en los intervalos de tiempo $[0,5]$, $(5,12)$, y $[13,23)$ son independientes, con distribuciones $\Pois(5\lambda), \Pois(7\lambda), \Pois(10\lambda)$ respectivamente.
\begin{minipage}{\linewidth}
            \centering
\includegraphics[width=2in]{figures/pp.pdf}
        \end{minipage}

\item[Dualidad Count-Time] Considera un proceso Poisson de emails llegando a una dirección de correo electrónico a una tasa de $\lambda$ emails por hora. Sea $T_n$ la marca de tiempo de cuando llega el email número $n$ (relatico a un tiempo inicial de $T_0 = 0$), y sea $N_t$ el número de emails que llegan en el intervalo $[0,t]$.

Ahora, vamos a encontrar la distribución de $T_1$ (el tiempo en el que llega el primer email). Véase que el evento $T_1 > t$ es el evento de que tengas que esperar más de $t$ horas para obtener el primer email. Este evento es lo mismo que el evento $N_t = 0$, el cual es el evento de que no hay emails en las primeras $t$ horas. Entonces,
\[P(T_1 > t) = P(N_t = 0) = e^{-\lambda t} \longrightarrow P(T_1 \leq t) = 1 - e^{-\lambda t}\]
Por lo que tenemos que $T_1 \sim \Expo(\lambda)$. Por la propiedad de pérdida de memoria y razonamientos similares, los tiempos de llegada entre emails se distribuyen $\Expo(\lambda)$; es decir, las diferencias entre $T_n - T_{n-1)$ se distribuyen $\Expo(\lambda)$.
\end{description}

%\section{Order Statistics}\smallskip \hrule height 2pt \smallskip
%\begin{description}
%    \item[Definition] Let's say you have $n$ i.i.d.~r.v.s $X_1, X_2,\dots, X_n$. If you arrange them from smallest to largest, the $i$th element in that list is the $i$th order statistic, denoted $X_{(i)}$. So $X_{(1)}$ is the smallest in the list and $X_{(n)}$ is the largest in the list. \smallskip
%    
%     Note that the order statistics are \emph{dependent}, e.g., learning $X_{(4)} = 42$ gives us the information that $X_{(1)},X_{(2)},X_{(3)}$ are $\leq 42$ and $X_{(5)},X_{(6)},\dots,X_{(n)}$ are $\geq 42$.
%    \item[Distribution]  Taking $n$ i.i.d. random variables $X_1, X_2, \dots, X_n$ with CDF $F(x)$ and PDF $f(x)$, the CDF and PDF of $X_{(i)}$ are:
%        \[F_{X_{(i)}}(x) = P (X_{(i)} \leq x) = \sum_{k=i}^n {n \choose k} F(x)^k(1 - F(x))^{n - k}\]
%    \[f_{X_{(i)}}(x) = n{n - 1 \choose i - 1}F(x)^{i-1}(1 - F(x))^{n-i}f(x)\]
%%    \item[Universality of the Uniform] Let $X_1,X_2,\dots,X_n$ be i.i.d.~CRVs with CDF $F$, and let $U_j=F(X_j)$. By UoU, $U_1,U_2,\dots,U_n$ are i.i.d.~$\Unif(0,1)$. Since $F$ is increasing, $F(X_{(1)}) \leq F(X_{(2)}) \leq \dots \leq F(X_{(n)})$, so $U_{(j)} = F(X_{(j)})$.
%    \item[Uniform Order Statistics]  The $j$th order statistic of i.i.d.~$U_1,\dots,U_n \sim \Unif(0,1)$ is $U_{(j)} \sim \Beta(j, n - j + 1)$.
%\end{description}
%
%
\section{Esperanza Condicional}\smallskip \hrule height 2pt \smallskip
\begin{description}
    \item[Condicionado sobre un Evento] Ya sabemos calcular $E(Y|A)$; la esperanza de $Y$ dado que ocurrió el evento $A$. Un caso muy importante ocurre cuando $A$ es el evento $X=x$. Nótese que $E(Y|A)$ es un \emph{número}. Por ejemplo:
    \begin{itemize}
    \item La esperanza de lanzar un dado justo, dado que el resultado es un número primo, es $\frac{1}{3} \cdot 2 + \frac{1}{3} \cdot 3 + \frac{1}{3} \cdot 5 = \frac{10}{3}$.
    \item Sea $Y$ el número de éxitos en $10$ pruebas Bernoulli independientes con una probabilidad $p$ de éxito. Sea $A$ el evento de que las primeras 3 pruebas dan éxitos. Entonces
    $$E(Y|A) = 3 + 7p$$
    ya que el número de éxitos entre las últimas 7 pruebas se distribuye $\Bin(7,p)$.
    \item Sea $T \sim \Expo(1/10)$ el tiempo que debes de esperar hasta que llegue el camión. Dado que ya esperaste $t$ minutos, el tiempo adicional de espera es de $10$ minutos más, por la propiedad de pérdida de memoria. Esto es, $E(T|T>t) = t + 10$.
    \end{itemize}
\end{description}
        \scalebox{0.85}{
                        \setlength{\extrarowheight}{7pt}
            \begin{tabular}{ccc}
                  \textbf{$Y$ Discreta} & \textbf{$Y$ Contínua} \\
            \toprule
            $E(Y) = \sum_y yP(Y=y)$ & $E(Y) =\int_{-\infty}^\infty yf_Y(y)dy$ \\
           % $E(Y|X=x) = \sum_y yP(Y=y|X=x)$ & $E(Y|X=x) =\int_{-\infty}^\infty yf_{Y|X}(y|x)dy$ \\
            $E(Y|A) = \sum_y yP(Y=y|A)$ & $E(Y|A) = \int_{-\infty}^\infty yf(y|A)dy$ \\ 
            \bottomrule
            \end{tabular}
        }
        \medskip
        
\begin{description}
    \item[Condicionando a una Variable Aleatoria] También podemos calcular $E(Y|X)$; la esperanza de $Y$ dada la variable aleatoria $X$. Esto es una \emph{función de la variable aleatoria $X$}. Esto \emph{no} es un número, excepto en casos especiales, como cuando $X \independent Y$.

    Para calcular $E(Y|X)$, encuentra $E(Y|X=x)$, y sustituye $X$ en donde aparezca $x$. Por ejemplo:
    \begin{itemize}
        \item Si $E(Y|X=x) = x^3 + 5x$, entonces $E(Y|X) = X^3 + 5X$
        \item Sea $Y$ el número de éxitos en $10$ pruebas Bernoulli intepdendientes, con probailidad $p$ de éxito, y sea $X$ el número de éxitos en las primeras $3$ pruebas. Entonces, $E(Y|X) = X + 7p$.
        \item Sea $X \sim \N(0,1)$ y $Y = X^2$. Entonces, $E(Y|X=x) = x^2$ ya que, si sabemos que $X=x$, entonces abemos que $Y=X^2$. Además, $E(X|Y=y) = 0$ ya que, si sabemos que $Y=y$, entonces sabemos que $X = \pm \sqrt{y}$, con las mismas probabilidades (por simetría). Por lo tanto, $E(Y|X) = X^2, E(X|Y)=0$.
    \end{itemize} 
    
        \item[Propiedades de la Esperanza Condicional] \quad
    \begin{enumerate}
        \item $E(Y|X) = E(Y)$ si $X \independent Y$
        \item $E(h(X)W|X) = h(X)E(W|X)$ (\textbf{sacando lo que conocemos}) \\
        En particulas, $E(h(X)|X) = h(X)$.
        \item $E(E(Y|X)) = E(Y)$ (\textbf{Ley de Adán}, alias la Ley de la Esperanza Total)
    \end{enumerate}

    \item[Ley de Adán (Ley de la Esperanza Total)] también se puede escribir de una manera análoga al TPT.

    Para cualesquiera eventos $A_1, A_2. \dots, A_n$ que hacen particiones en el espacio muestral,
        \begin{align*}
        E(Y) &= E(Y|A_1)P(A_1) + \dots + E(Y|A_n)P(A_n)
    \end{align*}
    Y para le caso especial donde las particiones son $A, A^c$, se puede escribir así
        \begin{align*}
            E(Y) &= E(Y|A)P(A) + E(Y|A^c)P(A^c)
    \end{align*}

    \item[Ley de Eva (Ley de la Varianza Total)] \quad
    \[\var(Y) = E(\var(Y|X)) + \var(E(Y|X))\]
\end{description}


%\section{MVN, LLN, CLT}\smallskip \hrule height 2pt \smallskip
%
%\subsection{Law of Large Numbers (LLN)}
%Let $X_1, X_2, X_3 \dots$ be i.i.d.~with mean $\mu$. The \textbf{sample mean} is $$\bar{X}_n = \frac{X_1 + X_2 + X_3 + \dots + X_n}{n}$$ The \textbf{Law of Large Numbers} states that as $n \to \infty$, $\bar{X}_n \to \mu$ with probability $1$. For example, in flips of a coin with probability $p$ of Heads, let $X_j$ be the indicator of the $j$th flip being Heads.  Then LLN says the proportion of Heads converges to $p$ (with probability $1$).
%
%\subsection{Central Limit Theorem (CLT)}
%\subsubsection{Approximation using CLT}
%We use $\dot{\,\sim\,}$ to denote \emph{is approximately distributed}. We can use the \textbf{Central Limit Theorem} to approximate the distribution of a random variable $Y=X_1+X_2+\dots+X_n$ that is a sum of $n$ i.i.d. random variables $X_i$. Let  $E(Y) = \mu_Y$ and $\var(Y) = \sigma^2_Y$. The CLT says
%\[Y \dot{\,\sim\,} \N(\mu_Y, \sigma^2_Y)\]
%If the $X_i$ are i.i.d.~with mean $\mu_X$ and variance $\sigma^2_X$, then $\mu_Y = n \mu_X$ and $\sigma^2_Y = n \sigma^2_X$. For the sample mean $\bar{X}_n$, the CLT says
%\[ \bar{X}_n = \frac{1}{n}(X_1 + X_2 + \dots + X_n) \dot{\,\sim\,} \N(\mu_X, \sigma^2_X/n) \]
%
%
%\subsubsection{Asymptotic Distributions using CLT}
%
%We use $\xrightarrow{D}$ to denote \emph{converges in distribution to} as $n  \to \infty$. The CLT says that if we standardize the sum $X_1 + \dots + X_n$  then the distribution of the sum converges to $\N(0,1)$ as $n \to \infty$:
%\[\frac{1}{\sigma\sqrt{n}} (X_1 + \dots + X_n - n\mu_X) \xrightarrow{D} \N(0, 1)\]
%In other words, the CDF of the left-hand side goes to the standard Normal CDF, $\Phi$. In terms of the sample mean, the CLT says
%\[ \frac{\sqrt{n} (\bar{X}_n - \mu_X)}{\sigma_X} \xrightarrow{D} \N(0, 1)\]
%
%\section{Markov Chains}\smallskip \hrule height 2pt \smallskip
%
%\subsection{Definition}
%\begin{minipage}{\linewidth}
%            \centering
%\includegraphics[width=2.3in]{figures/chainA.pdf}
%        \end{minipage}
%A Markov chain is a random walk in a \textbf{state space}, which we will assume is finite, say $\{1, 2, \dots, M\}$. We let $X_t$ denote which element of the state space the walk is visiting at time $t$. The Markov chain is the sequence of random variables tracking where the walk is at all points in time, $X_0, X_1, X_2, \dots$. By definition, a Markov chain must satisfy the \textbf{Markov property}, which says that if you want to predict where the chain will be at a future time, if we know the present state then the entire past history is irrelevant.  \emph{Given the present, the past and future are conditionally independent}. In symbols,
%\[P(X_{n+1} = j | X_0 = i_0, X_1 = i_1, \dots, X_n = i) = P(X_{n+1} = j | X_n = i)\]
%\subsection{State Properties}
%A state is either recurrent or transient.
%\begin{itemize}
%\item If you start at a \textbf{recurrent state}, then you will always return back to that state at some point in the future.  \textmusicalnote \emph{You can check-out any time you like, but you can never leave.}  \textmusicalnote
%\item Otherwise you are at a \textbf{transient state}. There is some positive probability that once you leave you will never return. \textmusicalnote \emph{You don't have to go home, but you can't stay here.} \textmusicalnote
%\end{itemize}
%A state is either periodic or aperiodic.
%\begin{itemize}
%\item If you start at a \textbf{periodic state} of period $k$, then the GCD of  the possible numbers of steps it would take to return back is  $k>1$.
%\item Otherwise you are at an \textbf{aperiodic state}. The GCD of  the possible numbers of steps it would take to return back is 1.
%\end{itemize}
%
%
%\subsection{Transition Matrix}
%Let the state space be $\{1,2,\dots,M\}$. The transition matrix $Q$ is the $M \times M$ matrix where element $q_{ij}$ is the probability that the chain goes from state $i$ to state $j$ in one step:
%\[q_{ij} = P(X_{n+1} = j | X_n = i)\]
%
%To find the probability that the chain goes from state $i$ to state $j$ in exactly $m$ steps, take the $(i, j)$ element of $Q^m$.
%\[q^{(m)}_{ij} = P(X_{n+m} = j | X_n = i)\]
%If $X_0$ is distributed according to the row vector PMF $\vec{p}$, i.e., $p_j = P(X_0 = j)$, then the PMF of $X_n$ is $\vec{p}Q^n$.
%
%
%
%\subsection{Chain Properties}
%A chain is \textbf{irreducible} if you can get from anywhere to anywhere. If a chain (on a finite state space) is irreducible, then all of its states are recurrent. A chain is \textbf{periodic} if any of its states are periodic, and is \textbf{aperiodic} if none of its states are periodic. In an irreducible chain, all states have the same period. \medskip
%
%A chain is \textbf{reversible} with respect to $\vec{s}$ if $s_iq_{ij} = s_jq_{ji}$ for all $i, j$.  Examples of reversible chains include any chain with $q_{ij} = q_{ji}$, with $\vec{s} = (\frac{1}{M}, \frac{1}{M}, \dots, \frac{1}{M})$, and random walk on an undirected network.
%
%\subsection{Stationary Distribution}
%
%Let us say that the vector $\vec{s} = (s_1, s_2, \dots, s_M)$ be a PMF  (written as a row vector). We will call $\vec{s}$ the \textbf{stationary distribution} for the chain if $\vec{s}Q = \vec{s}$. As a consequence, if $X_t$ has the stationary distribution, then all future $X_{t+1}, X_{t + 2}, \dots$ also have the stationary distribution. \\
%
%\smallskip
%
%For irreducible, aperiodic chains, the stationary distribution exists, is unique, and $s_i$ is the long-run probability of a chain being at state $i$. The expected number of steps to return to $i$ starting from $i$ is $1/s_i$.
%
%\smallskip
%
% To find the stationary distribution, you can solve the matrix equation $(Q' - I){\vec{s}\,}'= 0$. The stationary distribution is uniform if the columns of $Q$ sum to 1.
%
%\smallskip
%
%\textbf{Reversibility Condition Implies Stationarity}  If you have a PMF $\vec{s}$ and a Markov chain with transition matrix $Q$, then $s_iq_{ij} = s_jq_{ji}$ for all states $i, j$ implies that $\vec{s}$ is stationary.
%
%\subsection{Random Walk on an Undirected Network}
% \begin{minipage}{\linewidth}
%            \centering
%\includegraphics[width=1.6in]{figures/network1.pdf}
%        \end{minipage}
%\medskip
%
%If you have a collection of \textbf{nodes}, pairs of which can be connected by undirected \textbf{edges}, and a Markov chain is run by going from the current node to a uniformly random node that is connected to it by an edge, then  this is a random walk on an undirected network. The stationary distribution of this chain is proportional to the \textbf{degree sequence} (this is the sequence of degrees, where the degree of a node is how many edges are attached to it). For example, the stationary distribution of random walk on the network shown above is proportional to $(3,3,2,4,2)$, so it's $(\frac{3}{14}, \frac{3}{14}, \frac{2}{14}, \frac{4}{14}, \frac{2}{14})$. 
%
%\section{Continuous Distributions}\smallskip \hrule height 2pt \smallskip
%
\section{Distribuciones Discretas} \smallskip \hrule height 2pt \smallskip

\subsection{Distribuciones para cuatro formas de muestreo}
\begin{center}
    \begin{tabular}{ccc}
        ~ & \textbf{Reemplazo} & \textbf{Sin reemplazo}  \\
    \midrule
        \textbf{\# de experimentos fijo ($n$)} & Binomial & HGeom \\ 
        ~ & (Bern si $n = 1$) & ~ \\ 
        \textbf{Saca hasta el éxito $r$} & NBin & NHGeom \\ 
        ~ & (Geom si $r = 1$) & ~\\ \bottomrule
    \end{tabular}
\end{center}

\subsection{Distribución Bernoulli} La distribución Bernoulli es el casi más sencillo de la distribución Binomial, donde solo tenemos un experimento ($n=1$). Digamos que $X$ se distribuye $\Bern(p)$. Entonces sabemos que:
\begin{description}
    \item[Story] Un experimento se conduce un una probabilidad $p$ de "éxito", y $X$ es el indicator de ese éxito: $1$ significa éxito, $0$ significa fracaso.
    \item[Ejemplo] Sea $X$ la v.a. indicadora de Caras para un lanzamiento de una moneda justa. Entonces, $X \sim \Bern(1/2)$. Además, $1 - X \sim \Bern(1/2)$ es la v.a. indicadora de Sello.
    \item[FMP] 
    \[P(X = x) = p^x(1-p)^{1-x}\]
    o simplemente
    \[P(X = x) = \begin{cases} p, & x = 1 \\ 1-p, & x = 0 \end{cases}\]
\end{description}

\subsection{Distribución Binomial} 
\begin{minipage}{\linewidth}
            \centering
\includegraphics[width=2in]{figures/Bin10_05.pdf}
        \end{minipage}

Digamos que $X$ se distribuye $\Bin(n,p)$. Entonces sabemos que:
\begin{description}
    \item[Story] $X$ es el número de "éxitos" que lograremos en $n$ experimentos independientes, donde cada experimento es un éxito o fallo, cada uno con una probabilidad de éxito de $p$. También podemos escribir a $X$ como la suma de múltiples v.a.s independientes $\Bern(p)$. 
    
    Sea $X \sim \Bin(n,p)$ y $X_i \sim \Bern(p)$, donde todas las Bernoullis son independientes. Entonces, 
        \[X = X_1 + X_2 + X_3 + \dots + X_n\]
    \item[Ejemplo] Si Jaime quiere hacer 10 tiradas a un aro de basketball y cada tirada, de forma independiente, tiene una probabilidad de $\frac{3}{4}$ de asestar, entonces el número de tiradas que asesta se distribuye $\Bin(10,\frac{3}{4})$.
    
    \item[FMP] 
    \[P(X = x) = {n  \choose x} p^x(1-p)^{n-x}\]

\item[Propiedades] Sea $X \sim \Bin(n,p), Y \sim \Bin(m,p)$ con $X \independent Y$.
\begin{itemize}
\item \textbf{Redefiniendo el éxito} $n-X \sim \Bin(n,1-p)$
\item \textbf{Suma} $X+Y \sim \Bin(n+m,p)$
\item \textbf{Condicional} $X|(X+Y=r) \sim \HGeom(n,m,r)$
 \item \textbf{Relación Binomial-Poisson} $\Bin(n, p)$ es aproximadamente  $\Pois(np)$ si $p$ es chica.
   \item \textbf{Relación Binomial-Normal} $\Bin(n, p)$ es aproximadamene $\N(np,np(1-p))$ si $n$ es grande y $p$ no se acerca a $0$ o $1$.
  \end{itemize}
\end{description}

\subsection{Distribución Geométrica} Digamos que $X$ se distribuye $\Geom(p)$. Entonces sabemos que:
\begin{description}
    \item[Story] $X$ es el número de "fracasos" que tendremos antes de obtener nuestro primer éxito. Nuestro éxito tiene una probabilidad $p$.
    \item[Ejemplo] Si cada pokébola que lanzamos tiene una probabilidad de $\frac{1}{10}$ de atrapar a Mew, el número de pokébolas fallidas se distribuirá $\Geom(\frac{1}{10})$.
    \item[FMP] Con $q = 1-p$, 
    \[P(X = k) = q^kp\]
\end{description}

\subsection{Distribución del Primer Éxito} Equivalente a la Geométrica, excepto que incluye el primer éxito en la cuenta. Esto es $1$ más el número de fracasos. Si $X \sim \textnormal(FS)(p)$, entonces $E(X) = 1/p$.

\subsection{Distribución Binomial Negativa} Digamos que $X$ se distribuye $\NBin(r, p)$. Entonces Sabemos que:
\begin{description}
    \item[Story] $X$ es el número de "fracasos" que tendremos antes de conseguir nuestro éxito número $r$. Nuestros éxitos tienen una probabilidad $p$.
    \item[Ejemplo] Ash le ordena a Pikachu que use su movimiento Thundershock, que le atina a su objetivo un 60\% de las veces y puede dejar inconsciente a un Raticate en 3 impactos. El número de movimientos fallidos antes de que Pikachu noquée a Raticate con Thundershock se distribuye $\NBin(3,0.6)$
    \item[FMP] Con $q = 1-p$, 
     \[P(X = n) = {n+r - 1 \choose r -1}p^rq^n\]
\end{description}

\subsection{Distribución Hípergeométrica} Digamos que $X$ se distribuye $\Hypergeometric(w, b, n)$. Entonces sabemos que:
\begin{description}
    \item[Story] En una población de $w$ objetos deseados y $b$ objetos no deseados, $X$ será es el número de "éxitos" que tengamos al escoger $n$ objetos, sin reemplazo (léase, $X$ es el número de objetos deseados $w$ que saquemos). Se asume que la selección de los $n$ objetos es una \textbf{muestra aleatoria simple} (todos los conjuntos de $n$ objetos son equiprobables de ser escogidos).
    \item[Ejemplos] Está rebuscada la Story, pero hay varios ejemplos:
    \begin{itemize}
    \item Estás en el evento para seleccionar jóvenes para hacer el servicio militar. Para seleccionar a los cadetes, los militares meten $w$ bolas blancas y $b$ bolas negras en una tómbola, de las cuales sacarán $n$ bolas sin reemplazo (las bolas que sacan no las van a volver a meter a la tómbola). Si un potencial cadete saca una bola negra, tendrá que hacer el servicio militar. Se libra de hacerlo si le sale una negra. El número de jóvenes que tendrán que hacer el servicio es $\HGeom(w,b,n)$; el número que se salvan es de $\HGeom(b,w,n)$.
    \item Digamos que vamos a Viridian Forest con el objetivo de atrapar Pikachus. Ahí hay dos tipos de Pokémones: Pikachus y Weedles. Digamos que hay $b$ Weedles (no deseados) y $w$ Pikachus (éxitos) en el bosque. Entonces, encontraremos $n$ Pokémones en el bosque, y $X$ será el número de Pikachus que nos encontremos.
    \item El número de Aces en una mano de $5$ cartas.
    \item \textbf{Captura-recaptura} Un bosque tiene $N$ venados. Capturas $n$ de ellos, les pones un tag, y los sueltas. Después, vuelves a capturar una nueva muestra de tamaño $m$. El número de venados con tag en la nueva muestra es de $\HGeom(n,N-n,m)$.
    \end{itemize} 
    \item[FMP]
    \[P(X = k) = \frac{{w \choose k}{b \choose n-k}}{{w + b \choose n}}\]
\end{description}

\subsection{Distribución Poisson} Digamos que $X$ se distribuye $\Pois(\lambda)$. Entonces sabemos lo siguiente:
\begin{description}
    \item[Story] Hay eventos poco comunes (que tienen baja probabilidad de ocurrir) que pueden ocurrir de muchas formas diferentes. Estos eventos acontecen a una tasa promedio de $\lambda$ veces por unidad de espacio o tiempo. El número de eventos que ocurren en esa unidad de espacio o tiempo es $X \sim \Pois(\lambda)$.

    \item[Ejemplo] Una intersección muy transitada tiene un promedio de 2 accidentes al mes. Ya que un accidente es un evento poco común que puede ocurrir de muchísimas formas diferentes, es razonable modelar el número de accidentes al mes en esa intersección como $\Pois(2)$. Entonces, el número de accidentes que ocurren en dos meses se distribuye $\Pois(4)$.
    
    \item[Propiedades]
Sea $X \sim \Pois(\lambda_1)$ y $Y \sim \Pois(\lambda_2)$, con $X \independent Y$.

\begin{enumerate}
    \item \textbf{Suma} $X + Y \sim \Pois(\lambda_1 + \lambda_2)$
    \item \textbf{Condicional} $X | (X + Y = n) \sim \Bin\left(n, \frac{\lambda_1}{\lambda_1 + \lambda_2}\right)$
    \item \textbf{Gallina-huevo} Si hay $Z \sim \Pois(\lambda)$ objetos y de forma aleatoria e independiente "aceptamos" cada uno con una probabilidad $p$, entonces el número de objetos aceptados es $Z_1 \sim \Pois(\lambda p)$, y el número de objetos rechazados es $Z_2 \sim \Pois(\lambda (1-p))$, y $Z_1 \independent Z_2$.
    
\end{enumerate}
    
     \item[FMP] 
        \[P(X = k) = \frac{e^{-\lambda}\lambda^k}{k!}\]
\end{description}


%\section{Multivariate Distributions} \smallskip \hrule height 2pt \smallskip
%
%
%\subsection{Multinomial Distribution}
%    Let us say that the vector $\vec{X} = (X_1, X_2, X_3, \dots, X_k) \sim \textnormal{Mult}_k(n, \vec{p})$  where $\vec{p} = (p_1, p_2, \dots, p_k)$.
%\begin{description}
%    \item[Story]  We have $n$ items, which can fall into any one of the $k$ buckets independently with the probabilities $\vec{p} = (p_1, p_2, \dots, p_k)$.
%    \item[Example]  Let us assume that every year, 100 students in the Harry Potter Universe are randomly and independently sorted into one of four houses with equal probability. The number of people in each of the houses is distributed $\Mult_4$(100, $\vec{p}$), where $\vec{p} = (0.25, 0.25, 0.25, 0.25)$.
%        Note that $X_1 + X_2 + \dots + X_4 = 100$, and they are dependent.
%    \item[Joint PMF]  For $n = n_1 + n_2 + \dots + n_k$,
%        \[P(\vec{X} = \vec{n}) =  \frac{n!}{n_1!n_2!\dots n_k!}p_1^{n_1}p_2^{n_2}\dots p_k^{n_k}\]
%    \item[Marginal PMF, Lumping, and Conditionals]
%    Marginally, $X_i \sim \Bin(n,p_i)$ since we can define ``success" to mean category $i$. If you lump together multiple categories in a Multinomial, then it is still Multinomial. For example, $X_i + X_j \sim \Bin(n, p_i + p_j)$ for $i \neq j$ since we can define ``success" to mean being in category $i$ or $j$. Similarly, if $k=6$ and we lump categories 1-2 and lump categories 3-5, then
%        \[ (X_1+X_2,X_3+X_4+X_5,X_6) \sim \Mult_3(n, (p_1+p_2, p_3 + p_4+p_5,p_6))\]
%        Conditioning on some $X_j$ also still gives a Multinomial:
%        \[X_1, \dots, X_{k-1} | X_k = n_k \sim \Mult_{k-1}\left(n - n_k, \left(\frac{p_1}{1 - p_k}, \dots, \frac{p_{k-1}}{1 - p_k}\right)\right)\]
%    \item[Variances and Covariances]  We have  $X_i \sim \Bin(n, p_i)$ marginally, so $\var(X_i) = np_i(1-p_i)$. Also, $\cov(X_i, X_j) = -np_ip_j$ for $i \neq j$.
%    \end{description}
%    
%\subsection{Multivariate Uniform Distribution}
%See the univariate Uniform for stories and examples. For the 2D Uniform on some region, probability is proportional to area. Every point in the support has equal density, of value $\frac{1}{\textnormal{area of region}}$. For the 3D Uniform, probability is proportional to volume.
%
%\subsection{Multivariate Normal (MVN) Distribution}
%A  vector $\vec{X} = (X_1, X_2, \dots, X_k)$ is  Multivariate Normal if every linear combination is Normally distributed, i.e., $t_1X_1 + t_2X_2 + \dots + t_kX_k$ is Normal for any constants $t_1, t_2, \dots, t_k$. The parameters of the Multivariate Normal are the \textbf{mean vector} $\vec{\mu} = (\mu_1, \mu_2, \dots, \mu_k)$ and the \textbf{covariance matrix} where the $(i, j)$ entry is $\cov(X_i, X_j)$. 
%
%\begin{description}
%\item[Properties] The Multivariate Normal has the following properties.
%\begin{itemize}
%\item Any subvector is also MVN.
%\item If any two elements within an MVN are uncorrelated, then they are independent.
%\item The joint PDF of a Bivariate Normal $(X,Y)$ with $\N(0,1)$ marginal distributions and correlation $\rho \in (-1,1)$ is 
%$$  f_{X,Y}(x,y) = \frac{1}{2 \pi \tau} \exp{\left(-\frac{1}{2 \tau^2} (x^2+y^2-2 \rho xy)\right)},$$
%with $\tau = \sqrt{1-\rho^2}.$ 
%\end{itemize}
%\end{description}

\section{Distribuciones Contínuas Paramétricas}\smallskip \hrule height 2pt \smallskip

\subsection{Distribución Uniforme} Digamos que $U$ se distribuye $\Unif(a, b)$. Sabemos que:
\begin{description}
    \item[Propiedades] Para una distribución Uniforme, la probabilidad de sacar una muestra perteneciente a cualquier intérvalo del soporte es proporcional a la longitud de ese intérvalo. Véase \emph(Universalidad de la Uniforme} para otras propiedades.
    
    \item[Ejemplo] Guillermo es pésimo tirando dardos cuando juega, entonces sus dados caen uniformemente en todo su cuarto porque es igual de probable de que aparezcan en cualquier lado. Los dardos de Guillermo tienen una distribución uniforme sobre la superficie del cuarto. La Uniforme es la única distribución en donde la probabilidad de darle a cualquier región es proporcional a la longitud/área/volúmen de esa región, y donde la densidad de ocurrencias en cualquier punto en específico es constante a lo largo de todo su soporte.
    \item[FDP y FDA]  
    
 \begin{eqnarray*}
    f(x) = \left\{
      \begin{array}{lr}
        \frac{1}{b-a} & x \in [a, b] \\
        0 &  x \notin [a, b]
      \end{array}
    \right.
    %\hspace{.75 in}
    F(x) = \left\{
      \begin{array}{lr}
        0 & x < a \\
        \frac{x-a}{b-a} & x \in [a, b] \\
        1 &  x > b
      \end{array}
    \right. 
 \end{eqnarray*}

\end{description}

\subsection{Distribución Normal} Digamos que $X$ se distribuye $\N(\mu, \sigma^2)$. Entonces sabemos que:
\begin{description}
    \item[Story] Refiere a cualquier muestra en donde la mayoría de los valores se acumulan alrededor de una media/mediana/moda $\mu$, con la mitad de los mismos en un intérvalo entre $\mu \pm$ una desviación estándar de $\sigma$.
    \item[Ejemplos] En E.U.A., la altura promedio masculina es de $175$ cm., con una desviación estándar de $6.35$ cm. Dado que Estados Unidos tiene una población masculina de más de 169 millones de personas, es razonable modelar las alturas con una distribución Normal. Entonces, las alturas masculinas en E.U.A. se distribuyen $\N(175,6.35^2)$.
    \item[FDP] Nota: la FDA se calcula numéricamente; no tiene función
        \[f(x) = \frac{1}{\sigma \sqrt{2 \pi}} e^{\frac{-(x - \mu)^2}{2\sigma^2}}\] 
    %\item[Central Limit Theorem] The Normal distribution is ubiquitous because of the Central Limit Theorem, which states that the sample mean of i.i.d.~r.v.s will approach a Normal distribution as the sample size grows, regardless of the initial distribution.
    \item[Normal Estándar] La Normal Estándar, $Z \sim \N(0, 1)$, tiene una media $0$ y varianza $1$. Denotamos su FDP con $\Phi$.
    \item[Transformaciones] Cada vez que trasladamos una v.a. Normal (al agregarle una constante), o la escalamos (al multiplicarle por una constante), la cambiamos a otra v.a. Normal. Para cualquier $X \sim \N(\mu, \sigma^2)$, la podemos transformar en la normal estándar con la siguiente transformación:
    \[Z= \frac{X - \mu}{\sigma} \sim \N(0, 1) \]
\end{description}

\subsection{Distribución Exponencial}

Digamos que $X$ se distribuye $\Expo(\lambda)$. Sabemos lo siguiente:

\begin{description}

    \item[Story] $X$ es el tiempo de espera promedio entre las ocurrencias de eventos, los cuales ocurren a una tasa de $\lambda$ por unidad de tiempo/espacio.
    
    \item[Ejemplo] Estás sentado en un campo abierto a media noche, deseando que los aviones en el cielo fueran estrellas fugaces. Sabes que, en promedio, las estrellas fugaces pasan cada 15 minutos, pero una estrella fugaz no va a aparecer así nomás porque esperaste mucho tiempo. Tu tiempo de espera no tiene memoria; el tiempo adicional para que venga la siguiente estrella fugaz no depende de cuanto tiempo llevas esperando. Por lo tanto, las horas de espera hasta la siguiente estrella fugaz se distribuye $\Expo(4)$. Aquí, $\lambda=4$ es el \textbf{parámetro}, ya que las estrellas fugaces llegan, en promedio, $1$ casa $1/4$ de hora. El tiempo esperado hasta la siguiente estrella fugaz es, entonces, $1/\lambda = 1/4$ de hora.
    
    
    \item[Escalando la $\Expo(1)$]
        \[Y \sim \Expo(\lambda) \rightarrow X = \lambda Y \sim \Expo(1)\]
     
     \item[FDP y FDA] 
 \[f(x) = \lambda e^{-\lambda x}, x \in [0, \infty)\]
 
 \[F(x) = P(X \leq x) = 1 - e^{-\lambda x}, x \in [0, \infty)\]
    
    \item[Pérdida de Memoria] La distribución Exponencial es la única distribución contínua con pérdida de memoria. Esta propiedad dice que, para cualquier $X \sim \Expo(\lambda)$ y cualesquiera números positivos $s$ y $t$,
    
    \[P(X > s + t | X > s) = P(X > t)\]
    
    De forma equivalente,
    
    \[X - a | (X > a) \sim \Expo(\lambda)\]
    
    Por ejemplo, un producto con un tiempo de vida $\Expo(\lambda)$ siempre estará "como nuevo" (osea, no experimenta desgaste con el tiempo). Dado que el producto ha sobrevivido $a$ años, el tiempo adicional que durará sigue siendo $\Expo(\lambda)$.

    \item[Mínimo de Expos] Si tenemos v.a.s $X_i \sim \Expo(\lambda_i)$, entonces $\min(X_1, \dots, X_k) \sim \Expo(\lambda_1 + \lambda_2 + \dots + \lambda_k)$. 
    \item[Máximo de Expos] Si tenemos v.a.s ~$X_i \sim \Expo(\lambda)$, entonces $\max(X_1, \dots, X_k)$ tiene la misma distribución que $Y_1+Y_2+\dots+Y_k$, donde $Y_j \sim \Expo(j\lambda)$ y las $Y_j$ son independientes.     
\end{description}

\subsection{Distribución Gamma}
\begin{minipage}{\linewidth}
            \centering
\includegraphics[width=1.9in]{figures/gammapdfs.pdf}
        \end{minipage}
\medskip
Digamos que $X$ se distribuye $\Gam(a, \lambda)$. Entonces sabemos que:
\begin{description}
    \item[Story] $X$ es el tiempo de espera total para que ocurran $a$ número de eventos cuyo tiempo de espera se distribuye $\Expo(\lambda)$.
%    \item[Story] You sit waiting for shooting stars, where the waiting time for a star is distributed $\Expo(\lambda)$. You want to see $n$ shooting stars before you go home. The total waiting time for the $n$th shooting star is $\Gam(n,\lambda)$.
    \item[Ejemplo] Estás en un banco, y hay $3$ personas en frente de ti. El tiempo para atender a cada persona es $\Expo$, con una media de $2$ minutos. Sólo se puede atender a una persona a la ves. La distribución del tiempo que esperarás hasta que sea tu turno es de $\Gam(3,1/2)$ (ya que atienden a $1/2$ de personas cada minuto).
     \item[FDP] Nota: la FDA se calcula numéricamente; no hay función per se.
 \begin{eqnarray*}
 f(x) = \frac{1}{\Gamma(a)}(\lambda x)^ae^{-\lambda x}\frac{1}{x},
 \hspace{.1 in}
 x \in [0, \infty)
 \end{eqnarray*}


\end{description}

\subsection{Distribución Beta}
\begin{minipage}{\linewidth}
            \centering
\includegraphics[width=1.9in]{figures/Betapdfs.pdf}
        \end{minipage}
\medskip

\begin{description}

\item[Conjugado anterior a la Binomial] Desde la perspectiva Bayesiana de la estadística, se ve a los parámetros como variables aleatorias, para reflejar nuestra incertisumbre. El \emph{anterior} de un parámetro es su distribución antes de observar datos. Su \emph{posterior} es su distribución después de observar datos. Beta es el anterior de la Binomial, ya que si tienes un $p$ de una Binomial anterior que se distribuye Beta, entonces la distribución posterior de $p$ dados los datos de la Binomial también se distribuye Beta. Véase el siguiente modelo: 
    \begin{align*}
        X|p &\sim \Bin(n, p) \\
        p &\sim \Beta(a, b)
    \end{align*}
Entonces, después de observar $X = x$, obtenemos la distribución posterior.

\[p|(X=x) \sim \Beta(a + x, b + n - x) \]

\item[Relación Beta-Gamma] Si $X \sim \Gam(a, \lambda)$, $Y \sim \Gam(b, \lambda)$, con $X \independent Y$ entonces
    \begin{itemize}
    	\item $\frac{X}{X + Y} \sim \Beta(a, b)$
    	\item $X + Y \independent \frac{X}{X + Y}$
    \end{itemize}
    A esto se le conoce como el resultado del \textbf{banco-oficina postal}.
\end{description}


% \[E(X) = \frac{a}{\lambda},  Var(X) = \frac{a}{\lambda^2}\]
% \[X \sim G(a, \lambda),  Y \sim G(b, \lambda),  X \independent Y \rightarrow  X + Y \sim G(a + b, \lambda), \frac{X}{X + Y} \independent X + Y \]
% \[X \sim \Gam(a, \lambda) \rightarrow X = X_1 + X_2 + ... + X_a \textnormal{ for $X_i$ i.i.d. $\Expo(\lambda)$} \]
% \[\Gam(1, \lambda) \sim \Expo(\lambda) \]


%\subsection{Distribución $\chi^2$ (Ji cuadrada)}
%
%Digamos que $X$ se distribuye $\chi^2_n$. Entonces sabemos que:
%\begin{description}
%    \item[Story] $X$ es la suma de los cuadrados de $n$ v.a. Normales Estándar.
%    %\item[Example]  The sum of squared errors are distributed $\chi^2_n$
%     \item[FDP]
% \begin{eqnarray*}
% f(w) = \frac{1}{\sqrt{2\pi w}}e^{-w/2},
% w \in [0, \infty)
% \end{eqnarray*}
%    \item[Properties and Representations]
%\[X \textrm{ is distributed as } Z_1^2 + Z_2^2 + \dots + Z_n^2 \textrm{ for i.i.d.~$Z_i \sim \N(0,1)$}\]
%\[X \sim \Gam(n/2,1/2)\]
%\end{description}
\subsection{Distribución Cauchy}

Digamos que $X$ se distribuye $\Cauchy(\alpha, \beta)$. Entonces sabemos que:

\begin{description}
    \item[Story] $X$ se distribuye parecido a la Normal, acumulando valores alrededor de $\alpha$ con un rango intercuantiles definido por $\beta$.
    \item[FDP y FDA] Donde $\alpha \in [-\infty, \infty]$ y $\beta >0$
        \[f(x;\alpha, \beta) = \frac{1}{\pi \beta [1 + (\frac{x - \alpha}{\beta})^2]}\]
        \[F(x) = \frac{1}{2} + \frac{1}{\pi}\arctan(\frac{x - \alpha}{\beta})\]
    \item[No tiene Momentos finitos] No tiene Esperanzas, ni Varianzas, ni nada de eso.
\end{description}

\subsection{Distribución Weibull}

Digamos que $X$ se distribuye $\Wei(a,b)$. Entonces sabemos que:

\begin{description}
    \item[Story] $X$ como una distribución Exponencial elevada a una potencia, por lo que pierde la propiedad de pérdida de memoria, donde $a$ es el parámetro de forma y $b$ es el parámetro de escala.
    \item[Ejemplo] El tiempo de vida de una persona se puede modelar con Weibull.
    \item[FDP y FDA] Donde $a,b > 0$ y $x \geq 0$
        \[f(x;a,b) = abx^(b-1)e^{-ax^b}\] 
        \[F(x) = 1 - e^{-ax^b}\]
    \item[Relación Weibull-Exponencial] Si $b=1$, $X \sim \Expo(a)$.
\end{description}

\subsection{Distribución Pareto}

Digamos que $X$ se distribuye $\Par(x_0, \theta)$. Entonces sabemos que:

\begin{description}
    \item[Story] Se usa para modelar fenómenos en donde los valores se sesgan muchísimos a un lado. $x_0$ es el parámetro escala, mientras que $\theta$ el parámetro de forma.
    \item[Ejemplo] La riqueza de las personas se modela con Pareto, en el que el $80\%$ de la riqueza la acumula un $20\%$ de la población.
    \item[FDP y FDA] Donde $\theta,x_0 > 0$ y $x>x_0$
    \[f(x;x_0,\theta) = \frac{\theta}{x_0}(\frac{x_0}{x})^{\theta + 1}\]
    \[F(x) = 1 - (\frac{x_0}{x})^\theta\]
    \item[Función de supervivencia] Esta se define a partir de la FDA de Pareto, tal que:
    \[
    F(x) = \left\{\begin{array}{lr}
         (\frac{x_m}{x})^\theta & x \geq x_0, \\
         1 & x < 0
        \end{array}\right\}
    \]
\end{description}

\section{Propiedades de las Distribuciones} \smallskip \hrule height 2pt \smallskip
\subsection{FDAs Importantes}
\begin{description}
   \item[Normal Estándar] $\Phi$
    \item[Exponencial($\lambda$)] $F(x) = 1 - e^{-\lambda x}, \textrm{ para } x \in (0, \infty)$
    \item[Uniforme(0,1)] $F(x) = x, \textrm{ para } x \in (0, 1)$
\end{description}

%\subsection{Convolutions of Random Variables}
%A convolution of $n$ random variables is simply their sum. For the following results, let $X$ and $Y$ be \emph{independent}.
%\begin{enumerate}
%    \item  $X \sim \Pois(\lambda_1)$, $Y \sim \Pois(\lambda_2)$ $\longrightarrow X + Y \sim \Pois(\lambda_1 + \lambda_2)$
%    \item  $X \sim \Bin(n_1, p)$, $Y \sim \Bin(n_2, p)$ $\longrightarrow X + Y \sim \Bin(n_1 + n_2, p)$. $\Bin(n,p)$ can be thought of as a sum of i.i.d.~$\Bern(p)$ r.v.s.
%    \item  $X \sim \Gam(a_1, \lambda)$, $Y \sim \Gam(a_2, \lambda)$ $\longrightarrow  X + Y \sim\Gam(a_1 + a_2, \lambda)$.  $\Gam(n,\lambda)$ with $n$ an integer can  be thought of as a sum of i.i.d.~Expo($\lambda$) r.v.s.
%    \item  $X \sim \NBin(r_1, p)$, $Y \sim \NBin(r_2, p)$ $\longrightarrow  X + Y \sim\NBin(r_1 + r_2, p)$. $\NBin(r,p)$ can  be thought of as a sum of i.i.d.~Geom($p$) r.v.s.
%    \item  $X \sim \N(\mu_1, \sigma_1^2)$, $Y \sim \N(\mu_2, \sigma_2^2)$  $\longrightarrow  X + Y \sim \N(\mu_1 + \mu_2, \sigma_1^2 + \sigma_2^2)$
%\end{enumerate}

\subsection{Casos Especiales de las Distribuciones}
\begin{enumerate}
    \item $\Bin(1, p) \sim \Bern(p)$
    \item $\Beta(1, 1) \sim \Unif(0, 1)$
    \item $\Gam(1, \lambda) \sim \Expo(\lambda)$
    \item $\chi^2_n \sim \Gam\left(\frac{n}{2}, \frac{1}{2}\right)$
    \item $\NBin(1, p) \sim \Geom(p)$
\end{enumerate}

\subsection{Desigualdades}

\begin{enumerate}
\item \textbf{Cauchy-Schwarz} $|E(XY)| \leq \sqrt{E(X^2)E(Y^2)}$
\item \textbf{Markov} $P(X \geq a) \leq \frac{E|X|}{a}$ for $a>0$
\item \textbf{Chebyshev} $P(|X - \mu| \geq a) \leq \frac{\sigma^2}{a^2}$ for $E(X)=\mu, \var(X) = \sigma^2$
\item \textbf{Jensen} $E(g(X)) \geq g(E(X))$ for $g$ convex; reverse if $g$ is concave
\end{enumerate}


\section{Fórmulas Útiles} \smallskip \hrule height 2pt \smallskip
\subsection{Series Geométricas}
\[ 1 + r + r^2 + \dots + r^{n-1} = \sum_{k=0}^{n-1} r^k = \frac{1 - r^n}{1 -r} \]
\[ 1 + r + r^2 + \dots = \frac{1}{1-r} \textnormal{ if $|r|<1$} \]

\subsection{Función Exponencial ($e^x$)}
\[ e^x = \sum_{n=0}^\infty \frac{x^n}{n!}= 1 + x + \frac{x^2}{2!} + \frac{x^3}{3!} + \dots = \lim_{n \rightarrow \infty} \left( 1 + \frac{x}{n} \right)^n \]

\subsection{Fórmulas Gamma $\Gamma$ y Beta $\beta$}
Muchas veces puedes resolver integrales complicadas relacionandolas con las funciones Gamma y Beta:
\[ \Gamma(t) = \int_0^\infty x^{t-1}e^{-x}\, dx \hspace{1 cm} \beta(a,b) = \frac{\Gamma(a)\Gamma(b)}{\Gamma(a + b)} =  \int_0^1 x^{a - 1}(1-x)^{b-1}\, dx  \]
Recuerda que $\Gamma(a+1) = a \Gamma(a)$, y $\Gamma(n) = (n - 1)!$ si $n$ es un entero positivo.
\subsection{Aproximación de Euler para Sumas Harmónicas}
\[ 1 + \frac{1}{2} + \frac{1}{3} + \dots + \frac{1}{n} \approx \log n + 0.577 \dots\]
\subsection{Aproximación de Sterling para Factoriales}
\[ n! \approx \sqrt{2\pi n}\left(\frac{n}{e}\right)^n\]

\section{Definiciones Miscelaneas} \smallskip \hrule height 2pt \smallskip

\begin{description}
\item[Medianas y Cuantiles] Sea $X$ con una FDA $F$. Entonces, $X$ tiene una mediana $m$ si $F(m) \geq 0.5$, y $P(X \geq m) \geq 0.5$. Para $X$ Contínua, $m$ es tal que $F(m) = 1/2$. En general, el cuantil $a$ de $X$ es $\min \{x: F(x) \geq a\}$; la mediana es el caso en el que $a = 1/2$.
\item[log] Los estadísticos normalmente usan $\log$ para referirse al logaritmo natural (el base $e$).

\end{description}

%\newpage

%\section{Problemas de Ejemplo} \smallskip \hrule height 2pt \smallskip
%Contribuciones de Sebastian Chiu
%
%\subsection{Calculando Probabilidad}
%\begin{description}
%    \item[Problema] Un libro de texto tiene $n$ errores de ortografía, que se distribuyen de manera independiente en sus $n$ páginas. Eliges una página al azar. ¿Cuál es la probabilidad de que no tenga errores?
%    \item[Respuesta] Hay una probabilidad de $(1 - \frac{1}{n})$ de que un error en específico no esté en tu página, y por lo tanto una probabilidad de $\boxed{(1 - \frac{1}{n})^n}$ de que no haya faltas de ortografía en tu página. Si $n$ es muy grande, esto es aproximadamente $e^{-1}$.
%\end{description}
%
%\subsection{Linealidad e Indicadoras (1)} 
%\begin{description}
%    \item[Problema] En un grupo de $n$ personas, ¿cuál es el número esperado de cumpleaños distintos (día y mes)? ¿Cuál es el número esperado de cumpleaños que sean los mismos?
%    \item[Respuesta] Sea $X$ el número de de cumpleaños distintos, y sea $I_j$ la v.a. indicadora para el día número $j$. Recordemos que hay $365$ días en un año. Entonces,
%    \[E(I_j) = 1 - P(\textnormal{que nadie nazca en el día }j) = 1 - (364/365)^n\]
%
%    Por linealidad, $\boxed{E(X) = 365(1 - (364/365)^n)}$. Ahora, sea $Y$ el número de cumpleaños que son los mismos, y $J_i$ la v.a. indicadora de que el par número $i$ de personas tienen el mismo cumpleaños. La probabilidad de que cualquier par de personas en específico tengan el mismo cumpleaños es de $1/365$, entonces $\boxed{E(Y) = \binom{n}{2}/365}$
%\end{description}
%
%\subsection{Linearity and Indicators (2)}
%\emph{This problem is commonly known as the \textbf{hat-matching problem}}. There are $n$ people at a party, each with hat. At the end of the party, they each leave with a random hat. What is the expected number of people who leave with the right hat? \textbf{Answer:} Each hat has a $1/n$ chance of going to the right person. By linearity, the average number of hats that go to their owners is $\boxed{n(1/n) = 1}$.
%
%\subsection{Linearity and First Success}
%\emph{This problem is commonly known as the \textbf{coupon collector problem}}.
%There are $n$ coupon types. At each draw, you get a uniformly random coupon type. What is the expected number of coupons needed until you have a complete set? \textbf{Answer:} Let $N$ be the number of coupons needed; we want $E(N)$. Let $N = N_1 + \dots + N_n$, where $N_1$ is the draws to get our first new coupon, $N_2$ is the \emph{additional} draws needed to draw our second new coupon and so on. By the story of the First Success, $N_2 \sim \textnormal{FS}((n-1)/n)$ (after collecting first coupon type, there's $(n-1)/n$ chance you'll get something new). Similarly, $N_3 \sim \textnormal{FS}((n-2)/n)$, and $N_j \sim \textnormal{FS}((n-j+1)/n)$. By linearity,
%\[E(N) = E(N_1) + \dots + E(N_n) = \frac{n}{n} + \frac{n}{n-1} + \dots + \frac{n}{1} = \boxed{n\sum^n_{j=1} \frac{1}{j}}\]
%This is approximately $n (\log(n) + 0.577)$ by Euler's approximation.
%
%\subsection{Orderings of i.i.d. random variables}
%I call 2 UberX's and 3 Lyfts at the same time. If the time it takes for the rides to reach me are i.i.d., what is the probability that all the Lyfts will arrive first? \textbf{Answer:} Since the arrival times of the five cars are i.i.d., all $5!$ orderings of the arrivals are equally likely. There are $3!2!$ orderings that involve the Lyfts arriving first, so the probability that the Lyfts arrive first is $\boxed{\frac{3!2!}{5!} = 1/10}$. Alternatively, there are ${5 \choose 3}$ ways to choose 3 of the 5 slots for the Lyfts to occupy, where each of the choices are equally likely. One of these choices has all 3 of the Lyfts arriving first, so the probability is $\boxed{1 / {5 \choose 3} = 1/10}$.
%
%\subsection{Expectation of Negative Hypergeometric}
%What is the expected number of cards that you draw before you pick your first Ace in a shuffled deck (not counting the Ace)?
%\textbf{Answer:} Consider a non-Ace. Denote this to be card $j$. Let $I_j$ be the indicator that card $j$ will be drawn before the first Ace. Note that $I_j=1$ says that  $j$ is before all 4 of the Aces in the deck. The probability that this occurs is $1/5$ by symmetry. Let $X$ be the number of cards drawn before the first Ace. Then $X = I_1 + I_2 + ... + I_{48}$, where each indicator corresponds to one of the 48 non-Aces. Thus, \[E(X) = E(I_1) + E(I_2) + ... + E(I_{48}) = 48/5 = \boxed{9.6}.\]
%
%\subsection{Minimum and Maximum of RVs}
%What is the CDF of the maximum of $n$ independent Unif(0,1) random variables? \textbf{Answer:} Note that for r.v.s $X_1,X_2,\dots,X_n$, 
%\[ P(\min(X_1, X_2, \dots, X_n) \geq a) = P(X_1 \geq a, X_2 \geq a, \dots, X_n \geq a) \] Similarly, \[ P(\max(X_1, X_2, \dots, X_n) \leq a) = P(X_1 \leq a, X_2 \leq a, \dots, X_n \leq a) \] We will use this principle to find the CDF of $U_{(n)}$, where $U_{(n)} = \max(U_1, U_2, \dots, U_n)$ and $U_i \sim \Unif(0, 1)$ are i.i.d.
%\begin{align*}
%P(\max(U_1, U_2, \dots, U_n) \leq a)
%&= P(U_1 \leq a, U_2 \leq a, \dots, U_n \leq a) \\
%&= P(U_1 \leq a)P(U_2 \leq a)\dots P(U_n \leq a) \\
%&= \boxed{a^n}
%\end{align*}
%for $0<a<1$ (and the CDF is $0$ for $a \leq 0$ and $1$ for $a \geq 1$).
%
%\subsection{Pattern-matching with $e^x$ Taylor series}
%For $X \sim \Pois(\lambda)$, find $\displaystyle E\bigg{(}\frac{1}{X+1}\bigg{)}$. \textbf{Answer:} By LOTUS,
%\[E\left(\frac{1}{X+1}\right) = \displaystyle\sum_{k=0}^\infty \frac{1}{k+1} \frac{e^{-\lambda}\lambda^k}{k!} = \frac{e^{-\lambda}}{\lambda}\sum_{k=0}^\infty \frac{\lambda^{k+1}}{(k+1)!} = \boxed{\frac{e^{-\lambda}}{\lambda}(e^\lambda-1)}\]
%
%\subsection{Adam's Law and Eve's Law}
%William really likes speedsolving Rubik's Cubes. But he's pretty bad at it, so sometimes he fails. On any given day, William will attempt $N \sim \Geom(s)$ Rubik's Cubes. Suppose each time, he has probability $p$ of solving the cube, independently. Let $T$ be the number of Rubik's Cubes he solves during a day. Find the mean and variance of $T$. \textbf{Answer:} Note that $T|N \sim \Bin(N,p)$. So by Adam's Law, \[E(T) = E(E(T|N)) = E(Np) = \boxed{\displaystyle\frac {p (1-s)}{s}}\] Similarly, by Eve's Law, we have that \begin{align*}
%    \var(T) &= E(\var(T|N)) + \var(E(T|N)) =  E(Np(1-p)) + \var(Np)\\&= \frac{p(1-p)(1-s)}{s} +  \frac{p^2(1-s)}{s^2} = \boxed{\frac{p(1-s)(p+s(1-p))}{s^2}}
%    \end{align*}
%
%%\subsection{MGF -- Distribution Matching}
%%(Continuing the Rubik's Cube question above) Find the MGF of $T$. What is the name of this distribution and its parameter(s)? \textbf{Answer:}
%%
%%    By Adam's Law, we have  \begin{align*} E(e^{tT}) &= E(E(e^{tT}|N)) =E((pe^t + q)^N) = s\sum_{n=0}^\infty(pe^t + 1-p)^n(1-s)^n \\&=\frac{s}{1-(1-s)(pe^t+1-p)} =\frac{s}{s+(1-s)p-(1-s)pe^t}
%%    \end{align*}
%%    
%%    Intuitively, we would expect that $T$ is distributed Geometrically since $T$ is just a filtered version of $N$, which itself is Geometrically distributed. The MGF of $X\sim\Geom(\theta)$ is \[E(e^{tX}) = \displaystyle\frac{\theta}{1-(1-\theta) e^t}\]
%%    
%%    So, we would want to try to get our MGF into this form to identify what $\theta$ is. Taking our original MGF, it would appear that dividing by $s+(1-s)p$ would allow us to do this. Therefore, we have that \begin{center}
%%    $E(e^{tT}) = \displaystyle\frac{s}{s+(1-s)p - (1-s)pe^t} = \frac{\frac{s}{s+(1-s)p}}{1-\frac{(1-s)p}{s+(1-s)p}e^t}$
%%    \end{center}
%%    
%%    By pattern-matching, it thus follows that $\boxed{T \sim \Geom(\theta)}$ where \[\boxed{\theta = \displaystyle\frac{s}{s+(1-s)p}}\]
%
%\subsection{MGF -- Finding Moments}
%Find $E(X^3)$ for $X \sim \Expo(\lambda)$ using the MGF of $X$. \textbf{Answer:} The MGF of an $\Expo(\lambda)$ is $M(t) = \frac{\lambda}{\lambda-t}$. To get the third moment, we can take the third derivative of the MGF and evaluate at $t=0$: 
%\[\boxed{E(X^3) = \frac{6}{\lambda^3}}\]
%But a much nicer way to use the MGF here is via pattern recognition: note that $M(t)$ looks like it came from a geometric series:
%\[\frac{1}{1-\frac{t}{\lambda}} = \sum^{\infty}_{n=0} \left(\frac{t}{\lambda}\right)^n = \sum^{\infty}_{n=0} \frac{n!}{\lambda^n} \frac{t^n}{n!}\] The coefficient of $\frac{t^n}{n!}$ here is the $n$th moment of $X$, so we have $E(X^n) = \frac{n!}{\lambda^n}$ for all nonnegative integers $n$. 
%
%%\subsection{Markov chains (1)}
%%Suppose $X_n$ is a two-state Markov chain with transition matrix \[
%%Q = \bordermatrix{~ & 0 & 1 \cr
%%                  0 & 1-\alpha & \alpha \cr
%%                  1 & \beta & 1-\beta \cr}
%%\] Find the stationary distribution $\vec{s} = (s_0, s_1)$ of $X_n$ by solving $\vec{s} Q = \vec{s}$, and show that the chain is reversible with respect to $\vec{s}$. \textbf{Answer:} The equation $\vec{s}Q = \vec{s}$ says  that \[s_0 = s_0(1-\alpha) + s_1 \beta \textnormal{ and } s_1 = s_0(\alpha) + s_0(1-\beta)\] By solving this system of linear equations, we have \[\boxed{\vec{s} = \displaystyle\bigg{(}\frac{\beta}{\alpha+\beta}, \frac{\alpha}{\alpha+\beta}\bigg{)}}\] To show that the chain is reversible with respect to $\vec{s}$, we must show $s_i q_{ij} = s_j q_{ji}$ for all $i, j$. This is done if we can show $s_0 q_{01} = s_1 q_{10}$. And indeed, \[s_0  q_{01}  = \frac{\alpha\beta}{\alpha+\beta} = s_1 q_{10}\]
%%
%%\subsection{Markov chains (2)}
%%William and Sebastian play a modified game of Settlers of Catan, where every turn they randomly move the robber (which starts on the center tile) to one of the adjacent hexagons.
%%
%%% hex grid code modified from http://tex.stackexchange.com/questions/61429/how-to-draw-a-hexagonal-grid-with-numbers-in-the-cells
%%\begin{center}
%%\begin{tikzpicture} [hexa/.style= {shape=regular polygon,regular polygon sides=6,minimum size=1cm, draw,inner sep=0,anchor=south,rotate=30}]
%%\foreach \j in {0,...,2}{%
%%\pgfmathsetmacro\end{2+\j} 
%%  \foreach \i in {0,...,\end}{%
%%  \node[hexa] (h\i;\j) at ({(\i-\j/2)*sin(60)},{\j*0.75}) {};}  }      
%%\node [circle,minimum size=1cm] at (h2;2) {Robber};
%%\foreach \j in {0,...,1}{%
%%  \pgfmathsetmacro\end{3-\j} 
%%  \foreach \i in {0,...,\end}{%
%%  \pgfmathtruncatemacro\k{\j+2}  
%%  \node[hexa] (h\i;\k) at ({(\i+\j/2-0.5)*sin(60)},{2.25+\j*0.75}) {};}  } 
%%\end{tikzpicture}
%%\end{center}
%%
%%\begin{enumerate}[label=(\alph*)]
%%    \item Is this Markov chain irreducible? Is it aperiodic? \textbf{Answer:} $\boxed{\textnormal{Yes to both.}}$ The Markov chain is irreducible because it can get from anywhere to anywhere else. The Markov chain is  aperiodic because the robber can return back to a square in $2, 3, 4, 5, \dots$ moves, and the GCD of those numbers is $1$.
%%        \item What is the stationary distribution of this Markov chain? \textbf{Answer:} Since this is a random walk on an undirected graph, the stationary distribution is proportional to the degree sequence. The degree for the corner pieces is 3, the degree for the edge pieces is 4, and the degree for the center pieces is 6. To normalize this degree sequence, we divide by its sum. The sum of the degrees is $6(3) + 6(4) + 7(6) = 84$. Thus the stationary probability of being on a corner is $3/84 = 1/28$, on an edge is $4/84 =  1/21$, and in the center is $6/84 = 1/14$.
%%    \item What fraction of the time will the robber be in the center tile in this game, in the long run? \textbf{Answer:} By the above, $\boxed{1/14}$.
%%    \item What is the expected amount of moves it will take for the robber to return to the center tile? \textbf{Answer:} Since this chain is irreducible and aperiodic, to get the expected time to return we can just invert the stationary probability. Thus on average it will take $\boxed{14}$ turns for the robber to return to the center tile.
%%\end{enumerate}
%
%
%\section{Problem-Solving Strategies} \smallskip \hrule height 2pt \smallskip
%
%Contributions from Jessy Hwang, Yuan Jiang, Yuqi Hou
%\begin{enumerate}
%\item \textbf{Getting started.} Start by \emph{defining relevant events and random variables}. (``Let $A$ be the event that I pick the fair coin''; ``Let $X$ be the number of successes.'') Clear notion is important for clear thinking! Then decide what it is that you're supposed to be finding, in terms of your notation (``I want to find $P(X=3|A)$''). Think about what type of object your answer should be (a number? A random variable? A PMF? A PDF?) and what it should be in terms of.
%
%\emph{Try simple and extreme cases}. To make an abstract experiment more concrete, try \emph{drawing a picture} or making up numbers that could have happened. Pattern recognition: does the structure of the problem resemble something we've seen before?
%\item \textbf{Calculating probability of an event.} Use counting principles if the naive definition of probability applies. Is the probability of the complement easier to find? Look for symmetries. Look for something to condition on, then apply Bayes' Rule or the Law of Total Probability. 
%\item \textbf{Finding the distribution of a random variable.} First make sure you need the full distribution not just the mean (see next item). Check the \emph{support} of the random variable: what values can it take on? Use this to rule out distributions that don't fit.  Is there a \emph{story} for one of the named distributions that fits the problem at hand?  Can you write the random variable as a function of an r.v. with a known distribution, say $Y = g(X)$?
%\item \textbf{Calculating expectation.} If it has a named distribution, check out the table of distributions. If it's a function of an r.v. with a named distribution, try LOTUS. If it's a count of something, try breaking it up into indicator r.v.s. If you can condition on something natural, consider using Adam's law. 
%\item \textbf{Calculating variance.} Consider independence, named distributions, and LOTUS. If it's a count of something, break it up into a sum of indicator r.v.s. If it's a sum, use properties of covariance. If you can condition on something natural, consider using Eve's Law.
%\item \textbf{Calculating $E(X^2)$.}  Do you already know $E(X)$ or $\var(X)$? Recall that $\var(X) = E(X^2) - (E(X))^2$. Otherwise try LOTUS.
%\item \textbf{Calculating covariance.} Use the properties of covariance. If you're trying to find the covariance between two components of a Multinomial distribution, $X_i, X_j$, then the covariance is $-np_ip_j$ for $i \neq j$.
%\item \textbf{Symmetry.} If $X_1,\dots,X_n$ are i.i.d., consider using symmetry.
%\item \textbf{Calculating probabilities of orderings.} Remember that all $n!$ ordering of i.i.d.~continuous random variables $X_1,\dots,X_n$ are equally likely.
%\item \textbf{Determining independence.} There are several equivalent definitions. Think about simple and extreme cases to see if you can find a counterexample.
%\item \textbf{Do a painful integral.} If your integral looks painful, see if you can write your integral in terms of a known PDF (like Gamma or Beta), and use the fact that PDFs integrate to $1$?
%\item \textbf{Before moving on.} Check some simple and extreme cases, check whether the answer seems plausible, check for biohazards.\end{enumerate}
%
%
%\section{Biohazards} \smallskip \hrule height 2pt \smallskip
%
%Contributions from Jessy Hwang
%
%\begin{enumerate} 
%\item \textbf{Don't misuse the naive definition of probability.}   When answering ``What is the probability that in a group of 3 people, no two have the same birth month?'', it is \emph{not} correct to treat the people as indistinguishable balls being placed into 12 boxes, since that assumes the list of birth months \{January, January, January\} is just as likely as the list \{January, April, June\}, even though the latter is six times more likely. \\ 
%\item \textbf{Don't confuse unconditional, conditional, and joint probabilities.}  In applying $P(A|B) = \frac{P(B|A)P(A)}{P(B)}$, it is \emph{not} correct to say ``$P(B) = 1$ because we know  $B$ happened''; $P(B)$ is the \emph{prior} probability of $B$. Don't confuse $P(A|B)$ with $P(A,B)$. \\
%\item \textbf{Don't assume independence without justification.}  In the matching problem, the probability that card 1 is a match and card 2 is a match is not $1/n^2$.  Binomial and Hypergeometric are often confused; the trials are independent in the Binomial story and dependent in the Hypergeometric story. \\
%\item \textbf{Don't forget to do sanity checks.} Probabilities must be between $0$ and $1$. Variances must be $\geq 0$. Supports must make sense. PMFs must sum to $1$. PDFs must integrate to $1$. \\
%\item \textbf{Don't confuse random variables, numbers, and events.}  Let $X$ be an r.v. Then $g(X)$ is an r.v. for any function $g$. In particular, $X^2$, $|X|$, $F(X)$, and $I_{X>3}$ are r.v.s. $P(X^2 < X | X \geq 0), E(X), \var(X), $ and $g(E(X))$ are numbers. $X = 2$ and $F(X) \geq -1$ are events. It does not make sense to write $\int_{-\infty}^\infty F(X) dx$, because $F(X)$ is a random variable. It does not make sense to write $P(X)$, because $X$ is not an event. \\
%\item \textbf{Don't confuse a random variable with its distribution.}  To get the PDF of $X^2$, you can't just square the PDF of $X$. The right way is to use transformations. To get the PDF of $X + Y$, you can't just add the PDF of $X$ and the PDF of $Y$. The right way is to compute the \hyperref[convolutions]{convolution}.
%\item \textbf{Don't pull non-linear functions out of expectations.} $E(g(X))$ does not equal $g(E(X))$ in general. The St. Petersburg paradox is an extreme example.  See also Jensen's inequality. The right way to find $E(g(X))$ is with \hyperref[lotus]{LOTUS}.
%\end{enumerate}

%\vspace{1.5in}
%
%\section{Distributions in R} \smallskip \hrule height 2pt \smallskip
%
%\bigskip
%
%\begin{center}
%\begin{tabular}{cc}
%\hline
%\textbf{Command} & \textbf{What it does} \\
%\hline
%\texttt{help(distributions)} & shows documentation on distributions\\
%\texttt{dbinom(k,n,p)} & PMF $P(X=k)$ for $X \sim \Bin(n,p)$\\
%\texttt{pbinom(x,n,p)} & CDF $P(X \leq x)$ for $X \sim \Bin(n,p)$\\
%\texttt{qbinom(a,n,p)} &   $a$th quantile for $X \sim \Bin(n,p)$\\
%\texttt{rbinom(r,n,p)} &  vector of $r$ i.i.d.~$\Bin(n,p)$ r.v.s\\
%\texttt{dgeom(k,p)} & PMF $P(X=k)$ for $X \sim \Geom(p)$\\
%\texttt{dhyper(k,w,b,n)} & PMF $P(X=k)$ for $X \sim \HGeom(w,b,n)$\\
%\texttt{dnbinom(k,r,p)} & PMF $P(X=k)$ for $X \sim \NBin(r,p)$\\
%\texttt{dpois(k,r)} & PMF $P(X=k)$ for $X \sim \Pois(r)$\\
%\texttt{dbeta(x,a,b)} & PDF $f(x)$ for $X \sim \Beta(a,b)$\\
%\texttt{dchisq(x,n)} & PDF $f(x)$ for $X \sim \chi^2_n$\\
%\texttt{dexp(x,b)} & PDF $f(x)$ for $X \sim \Expo(b)$\\
%\texttt{dgamma(x,a,r)} & PDF $f(x)$ for $X \sim \Gam(a,r)$\\
%\texttt{dlnorm(x,m,s)} & PDF $f(x)$ for $X \sim \mathcal{LN}(m,s^2)$\\
%\texttt{dnorm(x,m,s)} & PDF $f(x)$ for $X \sim \N(m,s^2)$\\
%\texttt{dt(x,n)} & PDF $f(x)$ for $X \sim t_n$\\
%\texttt{dunif(x,a,b)} & PDF $f(x)$ for $X \sim \Unif(a,b)$\\
%\hline
%\end{tabular}
%\end{center}
%
%The table above gives R commands for working with various named distributions. Commands analogous to \texttt{pbinom}, \texttt{qbinom}, and \texttt{rbinom} work for the other distributions in the table. For example, \texttt{pnorm}, \texttt{qnorm}, and \texttt{rnorm} can be used to get the CDF,  quantiles, and random generation for the Normal.  For the Multinomial, \texttt{dmultinom} can be used for calculating the joint PMF and \texttt{rmultinom} can be used for generating random vectors. For the Multivariate Normal, after installing and loading the \texttt{mvtnorm} package \texttt{dmvnorm} can be used for calculating the joint PDF and \texttt{rmvnorm} can be used for generating random vectors.
%
%\vspace{4in}


\section{Recursos Recomendados} \smallskip \hrule height 2pt \smallskip

\bigskip

\begin{itemize}
\item Libro Introduction to Probability (\url{http://bit.ly/introprobability})
\item Stat 110 Online (\url{http://stat110.net})
\item Stat 110 Blog de Quora (\url{https://stat110.quora.com/})
\item FAQ de Probabilidad de Quora (\url{http://bit.ly/probabilityfaq})
\item R Studio (\url{https://www.rstudio.com})
\item Apuntes del Dr. Barrios (\url{https://gente.itam.mx/ebarrios/docs/apuntes_CP1.pdf})
\item Cuaderno de Ejercicio del Dr. Barrios (\url{https://gente.itam.mx/ebarrios/docs/CuadernoEjerciciosProbabilidad.pdf})
%\item LaTeX File (\texttt{\href{https://github.com/wzchen/probability_cheatsheet}{github.com/wzchen/probability\_cheatsheet}})
\end{itemize}

\begin{center}\emph{¡Comparte esta cheatsheet con tus panas!} \url{http://wzchen.com/probability-cheatsheet}\end{center}

\end{multicols*}

\section{Tabla de Distribuciones} 


\begin{center}
\renewcommand{\arraystretch}{3.7}
\begin{tabular}{cccccc}
\textbf{Distribución} & \textbf{FMP/FDP y Soporte} & \textbf{Esperanza}  & \textbf{Varianza} & \textbf{FGM} & \textbf{Story}\\
\hline 
\shortstack{Bernoulli \\ \Bern($p$)} & \shortstack{$P(X=1) = p$ \\$ P(X=0) = q=1-p$} & $p$ & $pq$ & $q + pe^t$ & Un sólo experimento de éxito o fracaso.\\
\hline
\shortstack{Binomial \\ \Bin($n, p$)} & \shortstack{$P(X=k) = {n \choose k}p^k q^{n-k}$  \\ $k \in \{0, 1, 2, \dots n\}$}& $np$ & $npq$ & $(q + pe^t)^n$ & Número de éxitos en $n$ pruebas Bernoulli.\\
\hline
\shortstack{Geométrica \\ \Geom($p$)} & \shortstack{$P(X=k) = q^kp$  \\ $k \in \{$0, 1, 2, \dots $\}$}& $q/p$ & $q/p^2$ & $\frac{p}{1-qe^t}, \, qe^t < 1$ & Número de fallos antes de obtener el primer éxito.\\
\hline
\shortstack{Binomial Negativa \\ \NBin($r, p$)} & \shortstack{$P(X=n) = {r + n - 1 \choose r -1}p^rq^n$ \\ $n \in \{$0, 1, 2, \dots $\}$} & $rq/p$ & $rq/p^2$ &  $(\frac{p}{1-qe^t})^r, \, qe^t < 1$ & Número de fallos antes de tener el éxito $r$\\
\hline
\shortstack{Hípergeométrica \\ \Hypergeometric($w, b, n$)} & \shortstack{$P(X=k) = \sfrac{{w \choose k}{b \choose n-k}}{{w + b \choose n}}$ \\ $k \in \{0, 1, 2, \dots,  n\}$} & $\mu = \frac{nw}{b+w}$ &$\left(\frac{w+b-n}{w+b-1} \right) n\frac{\mu}{n}(1 - \frac{\mu}{n})$& difícil & Número de veces que agarramos $w$ objetos deseados al sacar $n$ objetos.\\
\hline
\shortstack{Poisson \\ \Pois($\lambda$)} & \shortstack{$P(X=k) = \frac{e^{-\lambda}\lambda^k}{k!}$ \\ $k \in \{$0, 1, 2, \dots $\}$} & $\lambda$ & $\lambda$ & $e^{\lambda(e^t-1)}$ & Ocurrencias de un evento en cierto intérvalo de espacio/tiempo.\\
\hline
\hline
\shortstack{Uniforme \\ \Unif($a, b$)} & \shortstack{$ f(x) = \frac{1}{b-a}$ \\$ x \in (a, b) $} & $\frac{a+b}{2}$ & $\frac{(b-a)^2}{12}$ &  $\frac{e^{tb}-e^{ta}}{t(b-a)}$ & Todos los valores son equiprobables.\\
\hline
\shortstack{Normal \\ $\N(\mu, \sigma^2)$} & \shortstack{$f(x) = \frac{1}{\sigma \sqrt{2\pi}} e^{-\sfrac{(x - \mu)^2}{(2 \sigma^2)}}$ \\ $x \in (-\infty, \infty)$} & $\mu$  & $\sigma^2$ & $e^{t\mu + \frac{\sigma^2t^2}{2}}$ & Los valores se concentran en un centro.\\
\hline
\shortstack{Exponencial \\ $\Expo(\lambda)$} & \shortstack{$f(x) = \lambda e^{-\lambda x}$\\$ x \in (0, \infty)$} & $\frac{1}{\lambda}$  & $\frac{1}{\lambda^2}$ & $\frac{\lambda}{\lambda - t}, \, t < \lambda$ & Tiempo transcurrido entre ocurrencias de un evento.\\
\hline
\shortstack{Gamma \\ $\Gam(a, \lambda)$} & \shortstack{$f(x) = \frac{1}{\Gamma(a)}(\lambda x)^ae^{-\lambda x}\frac{1}{x}$\\$ x \in (0, \infty)$} & $\frac{a}{\lambda}$  & $\frac{a}{\lambda^2}$ & $\left(\frac{\lambda}{\lambda - t}\right)^a, \, t < \lambda$ & Tiempo transcurrido hasta ver la $n$ ocurrencia de un evento $\sim \Expo.$ \\
\hline
\shortstack{Beta \\ \Beta($a, b$)} & \shortstack{$f(x) = \frac{\Gamma(a+b)}{\Gamma(a)\Gamma(b)}x^{a-1}(1-x)^{b-1}$\\$x \in (0, 1) $} & $\mu = \frac{a}{a + b}$  & $\frac{\mu(1-\mu)}{(a + b + 1)}$ & difícil & Actualización de parámetros de una binomial.\\
\hline
\shortstack{Pareto} & \shortstack{$f(x;x_0, \theta) = \frac{\theta x_0^\theta}{x(\theta+1)}$ \\ $x > x_0, \theta >0$} & $\frac{\theta x_0}{\theta -1}, \theta > 1$  & $\frac{\theta x_0^2}{(\theta -1)^2(\theta -2)}, \theta >2$ & No existe & Los valores se sesgan mucho a un extremo.\\
\hline
\shortstack{Weibull} & \shortstack{$f(x;a,b) = abx^(b-1)e^{-ax^b}$ \\ $x \geq 0$} & $(\frac{1}{a})^{(1/b)} \Gamma(1+b^{-1})$  & $(\frac{1}{a})^{(1/b)} [\Gamma(1+2b^{-1} - \Gamma^2(1 - b^{-1})]$ & $a^{-t/b} \Gamma(1 + b^{-1})$ & Distribución Exponencial sin pérdida de memoria.\\
\hline
\shortstack{Cauchy} & $f(x;\alpha, \beta) = \frac{1}{\pi \beta [1 + (\frac{x - \alpha}{\beta})^2]}$ & No existe  & No tiene & No tiene & Como una Normal, pero más achicopalada\\
\hline
\end{tabular}
\end{center}


\end{document}
